
\chapter[Métodos de Desplazamientos para pórticos]{Métodos de Desplazamientos para pórticos}

% intro e hipotesis
En esta Unidad Temática se presentan métodos de análisis de estructuras planas formadas por barras en base a desplazamientos como incógnita principal. %
%
Este tema es presentado en los libros con dos grandes enfoques: por una parte un enfoque más clásico o analítico, basado en principios de superposición y razonamientos conceptuales, como se puede ver en \citep{Timoshenko1940a,CerveraRuiz2002ii}, y por otra parte, un enfoque más orientado a métodos numéricos para análisis estructural general, como se ve en \citep{Onate2013,Pilkey2002,Wunderlich2002}. %
%
Para maximizar la uniformidad de criterios en los temas abordados en el curso para el análisis de distintas estructuras, se opta por el segundo enfoque. %

En la Sección~\ref{sec:analiMatPort} las ecuaciones son derivadas utilizando los principios energéticos ya vistos en la Unidad Temática 2 y de forma similar a como es realizado en \citep{Reddy2002b}. %
%
Luego, en la Sección~\ref{sec:metSDPort}, las ecuaciones del método del clásico \textit{Slope-Deflection} (o Método de Equilibrio en \citep{CerveraRuiz2002ii}) son presentadas, como un caso particular del desarrollo general, en el cual se asume que la energía de deformación axial es despreciable respecto a la de flexión. %
%
Se invita al estudiante a leer los materiales complementarios para entender las diferencias de enfoque y ver también las similitudes en las ecuaciones obtenidas.
% ----------------------------




\section{Métodos de análisis matricial de pórticos} \label{sec:analiMatPort}

En esta sección se presenta el desarrollo de un método de análisis matricial de pórticos planos. %
%
Métodos basados en estos desarrollos están implementados en herramientas computacionales para el análisis de pórticos planos considerando deformación axial. %
Se utiliza un enfoque usualmente utilizado tanto en libros de resistencia de materiales \citep{Pilkey2002} como en libros orientados a métodos computacionales \citep{Onate2013}. %


\subsection{Hipótesis y definiciones fundamentales}

Para el análisis de pórticos es necesario abordar el estudio de vigas sometidas a cargas transversales y axiales, esto será llamado \textit{flexión compuesta}. %
%
Para esto, un posible camino es agregar el efecto de la directa a las ecuaciones de la teoría de vigas a flexión pura, presentadas en la Sección~\ref{sec:teovigastimo}. %
%
Sin embargo se optará por otro camino orientado a deducir las ecuaciones a partir de algunas hipótesis de la Teoría de Vigas integradas con la Teoría de la Elasticidad. %
%
Este enfoque es utilizado en literatura de referencia \citep{Wunderlich2002,Onate2013} y permite llegar a los mismos resultados.

\subsubsection{Hipótesis}
Se considera un sólido con una geometría como la mostrada en la Figura~\ref{fig:viga3d}, es decir una viga. Se considera que el campo de desplazamientos de los puntos de la viga, están dados por el vector formado por las funciones: $u(x,y,z)$, $v(x,y,z)$ y $w(x,y,z)$, representando desplazamientos en $x$, $y$ y $z$ respectivamente. %
%
Asumiendo que la flexión se produce en el plano $x-y$, se consideran las siguientes hipótesis:
%
\begin{enumerate}
	\item Los desplazamientos transversales (flecha) de todos los puntos en una sección transversal (ubicada en la coordenada $x$)  son pequeños e iguales al desplazamiento del eje de la viga.
	$$
	v(x,y,z) =  v(x).
	$$
	%
	\item Las secciones transversales permanecen planas y perpendiculares al eje deformado durante la deformación y los giros $\theta$ son pequeños. %
	%
	%
	Los desplazamientos axiales están, por lo tanto, dados por:
	\begin{equation} \label{eqn:despaxi}
	u(x,y,z) = u_G(x) - y \theta(x),
	\end{equation}
	donde $\theta$ es el ángulo que forma el vector tangente de la curva de la deformada del eje con la horizontal y $u_G(x)$ es la función del desplazamiento axial del baricentro de la sección ubicada en $x$. 
	%
	\item Todos los desplazamientos perpendiculares al plano de deformación de la viga son nulos
$$
w(x,y,z) = 0
$$
	%
	\item Se asume que no existen esfuerzos aplicados perpendiculares al plano de deformación de la viga
	\item Se desprecia la energía de deformación por cortante es decir, la distorsión angular.
\end{enumerate}

Respecto a la hipótesis de desplazamientos perpendiculares al plano, esta hipótesis representa una simplificación del comportamiento real de la estructura y el efecto de Poisson, sin embargo simplifica la aplicación de la ecuación constitutiva y el cálculo del tensor de deformaciones y permite llegar a las ecuaciones de la teoría de vigas de forma directa.

Respecto a la no consideración de energía de deformación por cortante, se recomienda al estudiante interesado consultar libros como \citep{Onate2013} donde se describen los elementos de viga de Timoshenko o  artículos recientes en los que se muestra la utilidad de este tipo de enfoques para simular el comportamiento real de estructuras \citep{Bui2014}. %
%
En este curso no se considerará deformación por cortante, hipótesis que puede ser razonable para vigas cuya relación entre largo y altura de sección transversal sea superior a 10: $\ell/h > 10$ (este número es adoptado como criterio para este curso, otros estudios numérico/experimentales pueden sugerir otros valores).

\subsubsection{Tensor de deformaciones}

Se comienza calculando las componentes del tensor de deformaciones aplicando la relación desplazamientos-deformación al campo de desplazamientos considerado (ver Ecuación~\eqref{eqn:despaxi}). %
%
La deformación axial  $\varepsilon_x$ está dada por:
%
\begin{equation}\label{eqn:expdef}
\varepsilon_x(x,y) =  \varepsilon_G (x) -y \frac{\partial \theta}{\partial x}(x).
\end{equation}
%
donde $\varepsilon_G$ es la deformación axial del eje de la viga, y está dada por
%
\begin{equation}
\varepsilon_G(x) =  \frac{\partial u_G}{\partial x} (x) .
\end{equation}


La distorsión angular $\gamma_{xy}$ está dada por:
\begin{equation}\label{eq:epsG}
  \gamma_{xy}(x,y) = -\theta(x) + \frac{\partial v}{\partial x} (x).
\end{equation}
%
La no dependencia de $\gamma_{xy}$ respecto a $y$ implica que la cara permanece plana. %
%
Al imponer que la distorsión angular (asociada a la deformación por cortante) sea nula, se aporta la condición:
\begin{equation}
\boxed{
\theta = \frac{\partial v}{\partial x},
}
\end{equation}
lo que es equivalente a que la normal a la sección transversal deformada coincida con la tangente a la curva deformada.

Finalmente la expresión de la deformación axial está dada por:
%
\begin{equation}\label{eqn:epsdef}
\boxed{
\varepsilon_x(x,y) =  \varepsilon_G (x) -y \frac{\partial^2 v}{\partial x^2}(x).
}
\end{equation}
lo que representa una extensión de la expresión obtenida en la Ecuación~\eqref{eqn:epstimo} para el caso con deformación axial.

\cajaactividad{
Demostrar que, considerando las hipótesis de desplazamiento mencionadas, la única componente del tensor de deformaciones no nula, es $\varepsilon_{xx}$.
}



\subsubsection{Tensor de tensiones}

Se considera que el material es elástico lineal e isótropo. Se utiliza la correspondiente ecuación constitutiva y el tensor de deformaciones obtenido, teniendo la relación:
%
\begin{equation}
  \sigma_x = E \varep_x,
\end{equation}
o también
\begin{equation}
	\sigma_x (x,y) =  E \varepsilon_G(x) - E y \frac{\partial^2 v}{\partial x^2} (x).
\end{equation}

Esta componente de $\bfsig$ es la única que produce trabajo interno en la expresión de la energía de deformación interna dada por la Ecuación~\eqref{eqn:energbarra}.

\subsection{Solicitaciones y convenciones de signo}


\subsubsection{Solicitaciones internas}

Se definen las solicitaciones internas correspondientes a la tensión axial: directa y momento. %
%
Para la directa se tiene
%
\begin{equation}
N (x) = \int_{A(x)} \sigma_x (x,y) \, \dif A
\end{equation}
%
usando la ecuación constitutiva y la expresión de la deformación axial se tiene:
%
\begin{equation}
N (x) = \int_{A(x)} \left( E \frac{\partial u_G}{\partial x}(x) - E y \frac{\partial^2 v}{\partial x^2}(x)   \right) \, \dif A.
\end{equation}
%
Usando que el vector $\bfe_x$ pasa por el punto $G$, baricentro de la sección transversal, se tiene
%
\begin{equation}\label{eqn:direc}
\boxed{
N (x) =  E A(x)  \varepsilon_G(x).
}
\end{equation}
%
Se puede destacar que la convención de signo de directa positiva a tracción es coherente con el resultado obtenido ya que en dicho caso $\varepsilon_G >0$.

Para el cálculo del momento según el eje $\bfe_z$ se considera la suma (integral) del momento de diferenciales de área por su brazo respectivo:
%
\begin{equation}
M_z (x) = \int_{A(x)} \left( y \bfe_y \,  \wedge \, \sigma_x (x,y) \bfe_x \right) \cdot \bfe_z \dif A.
\end{equation}
%
Sustituyendo las expresiones de las tensiones y la deformación dada por la Ecuación~\eqref{eqn:expdef} se obtiene:
%
\begin{equation}
M_z (x) = \int_{A(x)} y E \varep_{G} \left( \bfe_y \,  \wedge \, \bfe_x \right) \cdot \bfe_z \dif A + \int_{A(x)} -y^2 \frac{\partial^2 v}{\partial x^2} (x)  \left( \bfe_y \,  \wedge \, \bfe_x \right) \cdot \bfe_z \dif A.
\end{equation}

Calculando el producto mixto y usando que el primer momento de inercia respecto al baricentro es nulo, se obtiene:
%
\begin{equation}
M_z (x) = E \int_{A(x)} y^2 \dif A \,  \frac{\partial^2 v}{\partial x^2} (x)
\end{equation}
%
usando ahora como hipótesis que la barra tiene sección transversal uniforme $A(x)=A$ y definiendo el segundo momento de inercia respecto de $z$ como $I_z(x) = \int_{A} y^2 dA$ se obtiene:
%
\begin{equation}\label{eqn:momen}
\boxed{
M_z (x) = E I_z \frac{\partial^2 v}{\partial x^2}(x).
}
\end{equation}

Por simplicidad de notación, a continuación se omitirán los subíndices $z$ y el argumento $x$. %
%
Esta notación con subíndices volverá a ser utilizada al considerar problemas de flexión esviada.

La relación obtenida entre momento y curvatura es muy relevante para la caracterización del comportamiento de estructuras de vigas incluso cuando el comportamiento incluye fenómenos complejos como fisuración y otras no linealidades.


\subsubsection{Expresión de tensión axial en función de solicitaciones}


%
A partir de la expresiones de la directa, dada por la Ecuación~\eqref{eqn:direc}, y del momento, dado por la Ecuación~\eqref{eqn:momen}, se obtiene:
%
\begin{equation}
\varepsilon_G(x) = \frac{	N (x) }{E A}
\qquad
\frac{\partial^2 v}{\partial x^2}(x) = \frac{	M_z (x) }{E I_z} 
\end{equation}
%
Sustituyendo en la expresión de la deformación axial dada por la Ecuación~\eqref{eqn:epsdef} se obtiene:
%
\begin{equation}
\varepsilon_x(x,y) = \frac{	N (x) }{E A}  -y  \frac{ M_z (x) }{E I_z},
\end{equation}

y multiplicando por $E$ ambos miembros y usando la ecuación constitutiva se tiene
%
\begin{equation}
\boxed{
\sigma(x,y) = \frac{	N (x) }{A}
- y \frac{	M_z (x) }{I_z} 
}
\end{equation}


\subsubsection{Convenciones de signo}

Para el desarrollo de métodos de análisis de pórticos es útil y necesario definir una convención de signos diferente a la usada para las solicitaciones internas. %
%
En la \autoref{fig:conve} se muestran dos convenciones de signo a ser utilizadas en este documento y en todo el curso.
%
\begin{figure}[htb]
  \centering
  \def\svgwidth{0.8\textwidth}
  \input{figs/UT3/convenciones_signos_vigas.pdf_tex}
	\caption{Convenciones de signo para momentos nodales y cargas externas en coordenadas locales.}
	\label{fig:conve}
\end{figure}

Las \textbf{solicitaciones internas} definidas con la convención de signos usual para $\sigma$ (positivo tracción) corresponden a la convención de signos \textbf{1} de la figura. %
%
Esta convención corresponde a aquella en la cual momentos positivos representan una tracción de fibras inferiores.

Por otra parte, la convención \textbf{2} es útil para el desarrollo de métodos matriciales o computacionales, asociada a las \textbf{fuerzas externas} aplicadas.

\subsubsection{Ecuaciones de equilibrio}

Las ecuaciones para vigas sometidas a cargas transversales $q$ y axiales $b$ distribuidas por unidad de longitud están dadas por:
%
\begin{eqnarray}
\frac{dN}{dx}(x) & =& -b(x) \\
\frac{dV}{dx}(x) & =& q(x) \label{eqn:eqcortante}\\
\frac{\partial M}{\partial x}(x) & =& V(x)
\end{eqnarray}
%

La deducción de estas ecuaciones a partir del equilibrio de un segmento diferencial fue realizado en cursos anteriores. %
Por otra parte, este desarrollo también puede ser realizado a partir del teorema de trabajo virtual, de forma similar a como es hecho en la sección 5.4.5 de \citep{Hughes1987a}.


\subsection{Relaciones fuerzas-desplazamientos}\label{sec:mdvig}

Se considera un elemento de viga, formado por un material elástico lineal con módulo de Young $E$, con sección transversal uniforme de área $A$ e inercia $I$. El elemento de viga tiene largo $\ell$. %
%
Se considera sometido a fuerzas nodales axiales y transversales, así como también momentos nodales como se muestra en la Figura~\ref{fig:ejeviga}. %
%
En dicha figura se muestra el eje de la viga en la configuración de referencia con las fuerzas aplicadas y la configuración deformada con los desplazamientos y giros nodales indicados de acuerdo a la convención de signos 2 (ver Figura~\ref{fig:conve}). %
%
\begin{figure}[htb]
	\setlength{\unitlength}{0.8\textwidth}
	\centering
	\def\svgwidth{0.8\textwidth}
	\input{figs/UT4/eje_viga_deformada.pdf_tex}
	\caption{Esquemas de viga: (a) eje de viga en configuración de referencia (línea punteada) y eje de viga deformada (línea continua), (b) eje de viga de referencia con fuerzas aplicadas.}
	\label{fig:ejeviga}
\end{figure}

Por ahora se asume que no hay cargas aplicadas en el tramo intermedio de la viga, ni distribuidas ni puntuales. %
%
Se omitirá el sub-índice $z$ en $M_z$ y $\theta_z$. 

El desplazamiento en cada punto de la viga está dado por las funciones $u(x,y)$ y $v(x,y)$. %
Para deducir las relaciones esfuerzo desplazamiento se utilizará el primer teorema de Castigliano y se comienza planteando la expresión de la energía potencial de deformación en función de las funciones de desplazamiento. %

La energia potencial de deformación de la viga está dada por la expresión:
%
\begin{equation}
	\Pi_{int}(u,v) = \frac{1}{2} \int_{\Omega} E \varepsilon_x (x)^2 \dif V = \frac{1}{2} \int_{\ell} \int_{A} E \left( \varepsilon_G(x) -y \frac{\partial ^2 v}{\partial x^2}(x) \right)^2 \dif A \dif x .
\end{equation}

%
Desarrollando se tiene:
%
\begin{eqnarray}
	\Pi_{int}(u,v) 
	&=& \frac{1}{2} \int_{\ell} \int_{A} E \left( \varepsilon_G(x)  \right)^2 \dif A \dif x \nonumber\\
	\dots &-&             \int_{\ell} \int_{A} E y \varepsilon_G(x)  \frac{\partial ^2 v}{\partial x^2}(x)  \dif A \dif x \nonumber\\
	\dots &+&             \frac{1}{2} \int_{\ell} \int_{A} E y^2 \left(  \frac{\partial ^2 v}{\partial x^2}(x) \right)^2 \dif A \dif x .
\end{eqnarray}
%
Usando que el origen del sistema de coordenadas es el baricentro de la sección y siendo que la sección transversal es uniforme, se tiene que:
%
\begin{equation}\label{eqn:pivigatotal}
	\Pi_{int}(u,v)  =
	\underbrace{ \frac{1}{2} \int_{\ell} \int_{A} E \left( \varepsilon_G(x)  \right)^2 \dif A \dif x }_{\text{axial}}
	+  
	\underbrace{ \frac{1}{2} \int_{\ell} \int_{A} E y^2 \left(  \frac{\partial ^2 v}{\partial x^2}(x) \right)^2 \dif A \dif x}_{\text{flexión}}.
\end{equation}
%

En esta expresión fueron definidos los términos de energía de deformación por esfuerzo axial (dependiente de $u$)y por esfuerzo de flexión (dependiente de $v$). %
%
Esto permite desacoplar los efectos y hacer desarrollos independientes, por lo que a continuación se presentan expresiones de las energías de deformación axial y flexional, asumiendo cada efecto por separado.

\subsubsection{Interpolación en la deformación axial}

Considerando que no existen cargas axiales aplicadas en el tramo del elemento se tiene que la deformación axial del baricentro en una barra está dada por:
%
\begin{equation}
	\varepsilon_G(x) = \frac{u_2 - u_1}{\ell},
\end{equation}
%
a lo largo de todo el elemento, 
donde $u_1$ y $u_2$ son los desplazamientos axiales del nodo 1 y 2 respectivamente. %
Este resultado fue visto en la UT1 y puede considerarse que la interpolación del desplazamiento de cualquier punto intermedio es lineal. %

La energía de deformación por esfuerzos axiales puede ser escrita en función de los desplazamientos nodales como:
%
\begin{equation}\label{eqn:Uaxial}
	\boxed{
	\Pi_{int,\text{axial}}(u_1,u_2) = \frac{1}{2} \frac{E A}{\ell} (u_2 - u_1)^2 .
	}
\end{equation}
%



\subsubsection{Interpolación en la deformación a flexión}

Para el término de flexión, se asume que la viga se encuentra sometida a esfuerzos de flexión (es decir momentos y cortantes) y por lo tanto, integrando la ecuación de la elástica se obtiene que la flecha es una función de tercer grado, por lo que se considera la siguiente expresión polinómica general:
%
\begin{equation}\label{eqn:ecw}
	v(x) = a_3 x^3 + a_2 x^2 + a_1 x + a_0, \qquad \forall x \in [0,\ell].
\end{equation}

En lugar de trabajar con los parámetros $a_i$ se desea representar la función $v$ en función de los valores nodales de flecha y giro, es decir, $v_1$,  $\theta_1$, $v_2$ y $\theta_2$. %
%
Para esto, se plantean las siguientes relaciones entre la flecha $v$ evaluada en ciertos puntos en particular y los desplazamientos nodales:
%
\begin{equation}
	v_1 = v(0), \qquad
	\theta_1 = \frac{d v}{d x} (0), \qquad
	v_2 = v(\ell), \qquad
	\theta_2 = \frac{d v}{d x}(\ell).
\end{equation}

Resolviendo el sistema de ecuaciones lineales se obtiene:
%
\begin{eqnarray}
	a_3 &=& \frac{1}{\ell^3} \left( \theta_{1} \ell + \theta_{2} \ell + 2 v_{1} - 2 v_{2} \right) \nonumber\\
	a_2 &=&	\frac{1}{\ell^2} \left( - 2 \theta_{1} \ell - \theta_{2}  \ell - 	3 v_{1} + 3 v_{2}  \right) \nonumber\\
	a_1 &=&	\theta_{1} \nonumber\\
	a_0 &=&	v_{1}, \nonumber
\end{eqnarray}
%
y sustituyendo en la Ecuación~\eqref{eqn:ecw} se obtiene la expresión:
%
\begin{equation} \label{eqn:elastica}
	v(x) =  \varphi_{v_1}(x) v_1 + \varphi_{\theta_1}(x) \theta_1
	+ \varphi_{v_2}(x) v_2 + \varphi_{\theta_2}(x) \theta_2,
\end{equation}
%
donde las funciones $\varphi$ son funciones de interpolación dadas por:
%
\begin{eqnarray}
	\varphi_{v_1} (x) &=& \left(1 - \frac{3 x^{2}}{\ell^{2}} + \frac{2 x^{3}}{\ell^{3}}\right) \\
	\varphi_{\theta_1} (x) &=&	\left(x - \frac{2 x^{2}}{\ell} + \frac{x^{3}}{\ell^{2}}\right) \\
	\varphi_{v_2} (x) &=&	\left(\frac{3 x^{2}}{\ell^{2}} - \frac{2 x^{3}}{\ell^{3}}\right) \\
	\varphi_{\theta_2} (x) &=& \left(- \frac{x^{2}}{\ell} + \frac{x^{3}}{\ell^{2}}\right) .
\end{eqnarray}

En la \autoref{fig:phis} se muestran los gráficos de las funciones de interpolación, los cuales corresponden a las funciones de las elásticas para desplazamientos o giros unitarios de cada grado de libertad correspondiente. %
%


%
\begin{figure}[htb]
	\centering
	\subfloat[Funciones de interpolación $\varphi_{v_1}$ y $\varphi_{v_2}$.]{\includegraphics[width=0.47\textwidth]{phisv}\label{fig:phiv}}
	\subfloat[Funciones de interpolación $\varphi_{\theta_1}$ y $\varphi_{\theta_2}$.]{\includegraphics[width=0.47\textwidth]{phist}\label{fig:phit}}
	\caption{Gráfico de funciones de interpolación para $\ell=1$.}
	\label{fig:phis}
\end{figure}


Cualquier elástica $v$ dada por fuerzas y momentos nodales puede ser escrita como combinación de estas cuatro funciones de interpolación considerando como factores los valores de flecha y giro como se muestra en la Ecuación~\eqref{eqn:elastica}. %

Considerando esta interpolación en el término de energía de deformación a flexión de la Ecuación~\eqref{eqn:pivigatotal}, y utilizando que la sección transversal es uniforme en el elemento de viga se obtiene
%
\begin{equation} \label{eqn:Uflex}
	\boxed{
	\Pi_{int,\text{flexión}} (v_1,\theta_1,v_2,\theta_2) = \frac{1}{2} E I \int_{\ell} \left( \frac{\partial ^2 v}{\partial x^2}(x,v_1,\theta_1,v_2,\theta_2) \right)^2\dif x .
	}
\end{equation}



\subsubsection{Energía potencial de fuerzas externas}

La energía potencial de las fuerzas externas puede escribirse como:
\begin{equation}
	\boxed{
	\Pi_{ext} (\bfu) = - \bfu^T \bff
	}
\end{equation}
%
donde $\bff$ es el vector columna de las fuerzas nodales, dado por:
%
\begin{equation}
	\bff = [F_{x,1},F_{y,1},M_{1}, F_{x,2},F_{y,2},M_{2} ]^T.
\end{equation}
%
y $\bfu$ es el vector columna de las desplazamientos nodales, dado por:
%
\begin{equation}
	\bfu = [u_1,v_1,\theta_1,  u_2,v_2,\theta_2 ]^T.
\end{equation}

Es importante recordar que la convención de signos utilizada para fuerzas y momentos externos nodales es la número 2. %
%
Asimismo, esta convención también se corresponde con la utilizada para desplazamientos y giros nodales. %


\subsubsection{Aplicación de Teorema de Castigliano}

Se puede escribir por lo tanto la expresión de la energía potencial total de la viga:
%
\begin{equation}
	\Pi (\bfu) =  \Pi_{int,\text{axial}}(u_1,u_2) + \Pi_{int,\text{flexión}}(v_1,\theta_1,v_2,\theta_2) - \bfu^T \bff
\end{equation}
%
y aplicar las seis condiciones dadas por el primer teorema de Castigliano. Estas condiciones se pueden agrupar en, por un lado las asociadas a los grados de libertad $u_1$ y $u_2$:
%
\begin{eqnarray}
	\frac{\partial \Pi_{int,\text{axial}} }{\partial u_1}(u_1, u_2)  &=& F_{x,1} \\
	%
	\frac{\partial  \Pi_{int,\text{axial}} }{\partial u_2}(u_1, u_2)  &=& F_{x,2} 
\end{eqnarray}
y las condiciones asociadas a los grados de libertad $v_1$, $\theta_1$, $v_2$ y $\theta_2$:
%
\begin{eqnarray}
	\frac{\partial \Pi_{int,\text{flexión}} }{\partial v_1}(v_1,\theta_1,v_2,\theta_2) &=& F_{y,1} \\
	%
	\frac{\partial \Pi_{int,\text{flexión}} }{\partial \theta_1}(v_1,\theta_1,v_2,\theta_2)  &=& M_1 \label{eqn:ut3castm1} \\
	\frac{\partial \Pi_{int,\text{flexión}} }{\partial v_2}(v_1,\theta_1,v_2,\theta_2)  &=& F_{y,2} \\
	%
	\frac{\partial \Pi_{int,\text{flexión}} }{\partial \theta_2}(v_1,\theta_1,v_2,\theta_2)  &=& M_2.
\end{eqnarray}


Las condiciones asociadas a $u_1$ y $u_2$ son equivalentes a relaciones ya vistas, por ejemplo en la Ecuación~\eqref{eqn:criticoBarra}.

Las condiciones asociadas a las otras variables requieren un desarrollo. %
%
A modo de ejemplo se presenta el desarrollo para la condición del momento del nodo izquierdo, es decir, la Ecuación~\eqref{eqn:ut3castm1}. %
%
Usando la interpolación de la flecha y calculando la derivada, se tiene:
%
\begin{equation}
	\frac{\partial \Pi_{int}}{\partial \theta_1} = \frac{1}{2} EI \int_0^{\ell} 2 \left( \frac{\partial^2 v}{\partial x^2} (v_1,\theta_1,v_2,\theta_2) \varphi_{\theta_1}'' \right) \dif x
\end{equation}
%
para evaluar la derivada de $v$ se utilizan las expresiones de las derivadas de las funciones de interpolación dadas por:
%
\begin{eqnarray}
	\varphi_{v_1}'' (x) = \left(- \frac{6}{\ell^{2}} + \frac{12 x}{\ell^{3}}\right) \qquad %
	\varphi_{\theta_1}'' (x) =	\left( - \frac{4}{\ell} + \frac{6 x }{\ell^{2}}\right) \\
	\varphi_{v_2}'' (x) =	\left(\frac{6}{\ell^{2}} - \frac{12 x}{\ell^{3}}\right) \qquad %
	\varphi_{\theta_2}'' (x) = \left(- \frac{2}{\ell} + \frac{6 x}{\ell^{2}}\right) .
\end{eqnarray}
%
sustituyendo $v$ por la expresión dada por las funciones de interpolación se obtiene:
%
\begin{equation}
	\frac{\partial \Pi_{int}}{\partial \theta_1} = K_{v_1,\theta_1} v_1 + K_{\theta_1,\theta_1} \theta_1 + K_{v_2,\theta_1} v_2  + K_{\theta_2,\theta_1} \theta_2
\end{equation}
%
donde los coeficientes $K$ están dados por las integrales:
%
\begin{eqnarray}
	K_{v_1,\theta_1} = EI \int_0^\ell \varphi_{v_1}'' \varphi_{\theta_1}'' \dif x \qquad
	%
	K_{\theta_1,\theta_1} = EI \int_0^\ell \varphi_{\theta_1}'' \varphi_{\theta_1}'' \dif x \\
	%
	K_{v_2,\theta_1} = EI \int_0^\ell \varphi_{v_2}'' \varphi_{\theta_1}'' \dif x \qquad 
	%
	K_{\theta_2,\theta_1} = EI\int_0^\ell \varphi_{\theta_2}'' \varphi_{\theta_1}'' \dif x 
\end{eqnarray}
%
Calculando las integrales se tiene:
%
\begin{eqnarray}
	K_{v_1,\theta_1} &=& EI \left(   \frac{24}{\ell^2} -  \frac{84}{2} \frac{1}{\ell^2} + \frac{72}{3} \frac{1}{\ell^2} \right) = EI \frac{6}{\ell^2} \\
	%
	K_{\theta_1,\theta_1} &=& EI \left(   \frac{16}{\ell^2} -  \frac{48}{2}\frac{1}{\ell^2} +\frac{36}{3} \frac{1}{\ell^2}  \right) \ell = EI \frac{4}{\ell}\\
	%
	K_{v_2,\theta_1} &=& EI \left(  - \frac{24}{\ell^2} + \frac{84}{2} \frac{1}{\ell^2} - \frac{72}{3} \frac{1}{\ell^2} \right) = -EI \frac{6}{\ell^2} \\
	%
	K_{\theta_2,\theta_1} &=& EI \left(   \frac{8}{\ell^2} -  \frac{36}{2} \frac{1}{\ell^2} + \frac{36}{3} \frac{1}{\ell^2}  \right)\ell = EI \frac{2}{\ell}
\end{eqnarray}

Se obtiene por lo tanto que la ecuación de Castigliano para el giro del primer nodo, dada por la Ecuación~\eqref{eqn:ut3castm1}, puede ser escrita como:
%
\begin{equation}
	EI \frac{6}{\ell^2} v_1 +  EI \frac{4}{\ell} \theta_1 - EI \frac{6}{\ell^2} v_2 +    EI \frac{2}{\ell} \theta_2 = M_1.
\end{equation}


Repitiendo el procedimiento para las otras condiciones de Castigliano para la flexión se llega a:
%
\begin{eqnarray}
	EI \left( \dfrac{12}{\ell^{3}} v_1 + \dfrac{6}{\ell^{2}} \theta_1 - \dfrac{12}{\ell^{3}} v_2 + \dfrac{6}{\ell^{2}} \theta_2 \right)  &=& F_{y,1} \\
	EI \left( \frac{6}{\ell^2} v_1 +  \frac{4}{\ell} \theta_1 - \frac{6}{\ell^2} v_2   + \frac{2}{\ell} \theta_2 \right) &=& M_1 \\
	EI \left( -\dfrac{12}{\ell^{3}} v_1 - \dfrac{6}{\ell^{2}} \theta_1 +\dfrac{12}{\ell^{3}} v_2 - \dfrac{6}{\ell^{2}} \theta_2 \right) &=& F_{y,2} \\
	EI \left( \dfrac{6}{\ell^{2}} v_1 + \dfrac{2}{\ell} \theta_1 - \dfrac{6}{\ell^{2}} v_2 + \dfrac{4}{\ell}  \theta_2 \right) &=& M_2,
\end{eqnarray}
%



Las seis condiciones de Castigliano establecen una relación lineal entre las fuerzas $\bff$ y los desplazamientos $\bfu$, la cual puede ser escrita en forma matricial como:
%
\begin{equation}\label{eqn:eqrig}
	\bfK \bfu = \bff,
\end{equation}
%
donde la matriz $\bfK$ es la llamada matriz de rigidez:
%
\begin{equation}
	\bfK = \left[
	\begin{matrix}
		\dfrac{EA}{\ell} & 0 & 0 & - \dfrac{EA}{\ell} & 0 & 0 \\[3mm]
		%
		0 &  12 \dfrac{EI}{\ell^{3}} & 6 \dfrac{EI}{\ell^{2}} &  0& -12 \dfrac{EI}{\ell^{3}} & 6 \dfrac{EI}{\ell^{2}} \\[3mm]
		0 &  6 \dfrac{EI}{\ell^{2}} & 4 \dfrac{EI}{\ell} & 0& -6 \dfrac{EI}{\ell^{2}} & 2 \dfrac{EI}{\ell} \\[3mm]
		-\dfrac{EA}{\ell} & 0 & 0 &  \dfrac{EA}{\ell} & 0 & 0 \\[3mm]
		%
		0 &  -12 \dfrac{EI}{\ell^{3}} & -6 \dfrac{EI}{\ell^{2}} &0&  12 \dfrac{EI}{\ell^{3}} & -6 \dfrac{EI}{\ell^{2}} \\[3mm]
		0 &  6 \dfrac{EI}{\ell^{2}} & 2 \dfrac{EI}{\ell} & 0 & -6 \dfrac{EI}{\ell^{2}} & 4 \dfrac{EI}{\ell} \\[3mm]
	\end{matrix}
	\right].
\end{equation}


Es importante destacar que la matriz de rigidez es simétrica, lo que puede ser justificado por el hecho de que la función de energía potencial total es una función de tipo $C^2$ y que la matríz $\bfK$ es la hesiana  de dicha función. %

Por otra parte también se destaca que la matriz $\bfK$ tiene valores propios nulos con sus correspondientes vectores propios asociados. %
(pertenecientes al subespacio asociado al valor propio cero $S_0$). %
%
Esto quiere decir que existen movimientos (dados por los vectores de desplazamientos en dicho subespacio) que se corresponden con fuerzas nodales nulas, por lo tanto estos movimientos no introducen deformaciones o tensiones en el elemento. %
%
Dicho de otra forma, estos movimientos serán los movimientos de cuerpo rígido de la viga. %
%
Estos movimientos son 3 por lo que es necesario eliminar tres grados de libertad (a través de las condiciones de contorno) para obtener una matriz reducida con valores propios positivos y por lo tanto invertible. %
%
De esta forma la solución del sistema es única.






\subsection{Cargas equivalentes} \label{sec:cargequiv}

El desarrollo presentado anteriormente es válido para cargas aplicadas en los nodos, o extremos de la barra, sin embargo, en el caso en que la barra tenga fuerzas externas transversales aplicadas entre los nodos es posible considerar fuerzas y momentos aplicados en los nodos extremos de la barra, de forma tal que el efecto sea equivalente. %
%
Estas fuerzas nodales son llamadas \textit{cargas equivalentes}. %

La energía potencial total tendrá una forma del tipo
$$
\Pi(\bfu) = \Pi_{int}(\bfu) + \Pi_{ext}^{nodal}(\bfu) + \Pi_{ext}^{tramo}(\bfu)
$$

Este método permite obtener valores exactos de giros y flechas nodales, sin embargo, es necesario integrar las ecuaciones de equilibrio para obtener los diagramas de solicitaciones  y elástica en el  interior del tramo.

A continuación se consideran dos casos importantes de cargas en el tramo, con un enfoque como el presentado en \citep{Onate2013} donde se puede encontrar un desarrollo más completo. %

\subsubsection{Carga distribuida}

En el caso de carga distribuida $q(x)$ (de acuerdo a la convención 2), se tiene que la energía potencial de esta fuerza distribuida está dada por:
%
\begin{equation}
	\Pi_{ext}^q =
	- \int_0^\ell q(x) v(x) \dif x.
\end{equation}
%
donde $q(x)$ tiene mismo sentido que $v(x)$.

El objetivo es entonces encontrar las fuerzas nodales equivalentes que tengan la misma energía potencial. Para esto se sustituye la expresión de $v(x)$ dada por la Ecuación~\eqref{eqn:elastica}, obteniendo:
%
\begin{equation}
	\Pi_{ext}^{eq} =
	-F_{y,1}^{eq} v_1 - M_1^{eq} \theta_1 -F_{y,2}^{eq} v_2 -  M_2^{eq} \theta_2
\end{equation}
donde
\begin{eqnarray}
	F_{y,1}^{eq} = \int_{0}^{\ell} q(x) \varphi_{v_1}(x) \dif x \qquad & \displaystyle M_1^{eq} = \int_{0}^{\ell}q(x) \varphi_{\theta_1}(x)  \dif x  \nonumber\\
	F_{y,2}^{eq} = \int_{0}^{\ell}q(x) \varphi_{v_2}(x)  \dif x \qquad & \displaystyle M_2^{eq} = \int_{0}^{\ell} q(x)\varphi_{\theta_2}(x)  \dif x  \nonumber
\end{eqnarray}

En el caso particular de carga distribuida uniforme $q(x) = q$ se tiene:
%
\begin{eqnarray}
	F_{y,1}^{eq} &=& \frac{q \ell}{2} \\
	M_1^{eq} &=& \frac{q \ell^{2}}{12} \\
	F_{y,2}^{eq} &=& \frac{q \ell}{2} \\
	M_2^{eq} &=& - \frac{q \ell^{2}}{12}
\end{eqnarray}

\subsubsection{Carga puntual}

Se considera ahora que la carga puntual $P$ es aplicada (según la convención 2) en el punto de coordenada $x= x_P$ con $x_P \in (0,\ell)$ y que $q(x)=0$.

La energía potencial de la fuerza en el tramo está dada por:
%
\begin{equation}
	\Pi_{ext}^{P} = - P v (x_P) =  -P \varphi_{v_1}(x_P) v_1 - P \varphi_{\theta_1}(x_P) \theta_1
	- P \varphi_{v_2}(x_P) v_2 - P \varphi_{\theta_2}(x_P) \theta_2,
\end{equation}
%
por lo tanto las fuerzas nodales equivalentes a la carga puntual en $x_P$ son calculadas evaluando las funciones $\varphi$. %
%
Por ejemplo el momento nodal en 1 es
%
\begin{equation}
	M_1^{eq} = P \varphi_{\theta_1} (x_P) = P  \left(x_P - \frac{2 x_P^{2}}{\ell} + \frac{x_P^{3}}{\ell^{2}}\right)
\end{equation}
%
expresión que, factorizando, puede ser reescrita como:
\begin{equation}\label{eqn:M1eqP}
	M_1^{eq} = P x_P (\ell-x_P)^2 \frac{1}{\ell^2}.
\end{equation}

Para el otro momento se tiene
\begin{equation}\label{eqn:M2eqP}
	M_2^{eq} = -P x_P^2 (\ell-x_P) \frac{1}{\ell^2},
\end{equation}
%
y para las fuerzas equivalentes se tiene
%
\begin{equation}\label{eqn:FyeqP}
	F_{y,1}^{eq} = P \left(1 - \frac{3 x^{2}}{\ell^{2}} + \frac{2 x^{3}}{\ell^{3}}\right) \qquad
	F_{y,2}^{eq} =
	P \left(\frac{3 x^{2}}{\ell^{2}} - \frac{2 x^{3}}{\ell^{3}}\right).
\end{equation}




Las fuerzas externas equivalentes, tanto para el caso de carga distribuida como el de carga puntual, son sumadas al término de fuerzas externas nodales, obteniendo una ecuación de rigidez más general
%
\begin{equation}
	\bfK \bfu = \bff + \bff^{eq}.
\end{equation}




\subsection{Ensamblado de sistema y comentarios sobre MEF}

El último paso del estudio de este método para pórticos consiste en construir una matriz de rigidez para toda la estructura, en un sistema de coordenadas global que permita sumar las fuerzas y momentos en cada nodos o elemento de la misma.
%
Luego de esto, en caso de tener resortes se agregan, y finalmente se resuelve el sistema.

\subsubsection{Cambio de sistema de coordenadas}

Para encontrar la relación entre desplazamientos y fuerzas nodales en problemas de pórticos donde los ejes de los elementos que concurren a un nodo no coinciden, es útil definir un sistema de coordenadas global de la misma forma en que fue realizado para reticulados planos.
%

En el caso de pórticos el cambio de base se aplica a los desplazamientos nodales, mientras que los giros se mantienen en el mismo versor de referencia. %
%
La matriz de cambio de base esta dada por
%
\begin{equation}
	\bfR^e = 
	\left[
	\begin{matrix}
		c & -s & 0 & 0 &  0 & 0 \\
		s & c & 0 & 0 &  0 & 0\\
		0 & 0 &1 & 0 &  0 & 0\\
		0 & 0 &0  &   c & -s& 0\\
		0 & 0 &0  &  s & c & 0\\
		0 & 0 &0 & 0 &  0 & 1\\
	\end{matrix}
	\right]
	\qquad
	\bfu^e = \bfR^e \bfu^e_L
\end{equation}
%
donde $c=\cos(\alpha^e)$ y $s=\sin(\alpha^e)$ y $\alpha^e$ es el ángulo medido positivo antihorario desde el eje global al local. %
%
La Ecuación~\eqref{eqn:eqrig} fue desarrollada en coordenadas locales utilizando la notación $\bfu$, la cual en el caso de sistemas coordenadas locales y globales correspondería a los ejes locales, por lo que debería ser escrita como
\begin{equation}
	\bfK_L \bfu_L = \bfF_L.
\end{equation}
%
La matriz de rigidez en coordenadas globales está dada por:
\begin{equation}
	\bfK^e_G = \bfR^e \bfK^e_L (\bfR^e)^T
\end{equation}

%
El sistema global obtenido luego de realizado el ensamblado es denotado por:
\begin{equation}
	\bfK_G \bfu = \bfF_G.
\end{equation}
donde $\bfF_G$ es el vector de fuerzas externas nodales en coordenadas globales.

La determinación de desplazamientos a través del ensamblado de un sistema de ecuaciones lineales es llamado Análisis Matricial dado que se realiza a través del ensamblado e inversión de matrices de dimensiones considerables. %
%
La aplicación de este método para estructuras con cargas nodales es equivalente al MEF utilizando elementos de pórtico (\textit{frame} en inglés) \citep{Onate2013}, procedimiento utilizado por la mayoría de los programas computacionales de determinación de solicitaciones.



\subsubsection{Apoyos elásticos}

Para la resolución mediante análisis matricial o  de forma computacional, las fuerzas introducidas por los resortes pueden ser consideradas a través de una matriz de rigidez para cada elemento (en coordenadas globales):
\begin{equation}
	\bfK_S^e = 
	\left[
	\begin{matrix}
		k_{u1} & 0 & 0 & 0 &  0 & 0 \\
		0 & k_{v1} & 0 & 0 &  0 & 0\\
		0 & 0 &k_{\theta1} & 0 &  0 & 0\\
		0 & 0 &0  &   k_{u2} & 0& 0\\
		0 & 0 &0  &  0 & k_{v2} & 0\\
		0 & 0 &0 & 0 &  0 & k_{\theta2}\\
	\end{matrix}
	\right]
\end{equation}
que puede ser ensamblada y sumada a la matriz de rigidez global, obteniendo un sistema dado por
%
\begin{equation}
	\left(\bfK_G + \bfK_S \right) \bfu = \bfF_G.
\end{equation}




%\subsubsection{Método de los Elementos Finitos}
%
%El Método de Elementos Finitos (MEF) puede también ser considerado un método de desplazamientos ya que las incógnitas principales son desplazamientos y giros. %
%%
%Por otra parte, el enfoque del método permite obtener las ecuaciones que gobiernan la deformación de diversos elementos estructurales \citep{Onate2013}.

%Se pueden enumerar algunas diferencias centrales respecto a los métodos analíticos.
%%
%\begin{itemize}
%	\item Para cargas puntuales aplicadas en el tramo, en el MEF se suele dividir el elemento de viga en dos elementos, agregando un nodo en el punto de aplicación de la carga. Esto aumenta la cantidad de variables, lo cual no es relevante dado que el sistema lineal es resuelto numéricamente, y también simplifica la implementación del método.
%	%
%	\item Las cargas distribuidas suelen ser consideradas discretizando el elemento en un número apropiado de elementos finitos agregando nodos intermedios y por lo tanto más incógnitas del problema. De esta forma se pueden calcular los diagramas de solicitaciones aproximados de forma directa. % 
%	Se debe también tener en cuenta que las solicitaciones no necesariamente serán exactas dentro del dominio de cada elemento, dado que existen aproximaciones por la interpolación considerada.
%	%
%	\item En el MEF es posible automatizar la verificación de estabilidad del esquema básico estructural considerado, a través del análisis de los valores propios de la matriz de rigidez ensamblada y con las condiciones de contorno aplicadas. %
%	%
%	Esto permite automatizar esa verificación, considerada muy importante al obtener solicitaciones de estructuras de grandes dimensiones \citep{Kannan2014}. 
%\end{itemize}
%
%En este curso no se mencionan aspectos sobre la convergencia de las soluciones al aumentar la cantidad de elementos finitos, el estudiante interesado puede consultar \citep{Hughes1987a}.

%El ONSAS es un ejemplo de herramienta numérica basada en el método de los elementos finitos.








\cajaactividad{
	Calcular la flecha máxima de una viga de largo $\ell$, biempotrada, con rigidez flexional uniforme $E I$ y una carga puntual $P$ aplicada en la mitad de su longitud, usando el método de los desplazamientos en su forma matricial, utilizando dos elementos.
}




%\subsection{Ejemplo}

%\subsection{Principio de trabajos virtuales}

%El principio de trabajos virtuales establece que la elástica de equilibrio $w(x)$ será aquella que cumpla que
%%
%\begin{equation}
%EI \int_{0}^{L} \frac{d^2 w}{d x^2} \, \frac{d^2 \tilde{w} }{d x^2} \, \dif x = V_1 \tilde{w}_1 +  M_1 \tilde{\theta}_1 + V_2 \tilde{w}_2 + M_2 \tilde{\theta}_2
%\end{equation}
%






%
%\subsubsection{Comparación MSD, MEF y soluciones analíticas}
%
%Los métodos MSD y MEF son derivados a partir de las mismas ecuaciones y ambos son métodos de desplazamientos. %
%Sin embargo se puede decir que una diferencia a tener en cuenta es la cantidad de incógnitas a determinar. %
%%
%En el MSD las cargas en el tramo son incluidas a través del vector de fuerzas nodales mientras que en el MEF se suelen agregar nodos intermedios. %
%% 
%
%







\section{Métodos analíticos para pórticos} \label{sec:metSDPort}

Las ecuaciones desarrolladas en la sección anterior pueden ser simplificadas bajo ciertas hipótesis, para dar lugar a las ecuaciones de métodos analíticos como el Método de \textit{Slope-Deflection}.

La primera aproximación/hipótesis que se realiza es, asumir que la energía de deformación axial es despreciable respecto a la energía de deformación por flexión. %
Otra aproximación o convención utilizada es pasar de trabajar con dos flechas $v_2$ y $v_1$ a una única variable de giro relativo de la cuerda de la viga $\psi$, como se verá más adelante.

Se comienza por la hipótesis de energía de deformación.

\subsection{Comparación de energías de deformación}

La hipótesis de energía de deformación axial despreciable en el contexto de vigas sometidas a flexión compuesta, puede ser escrita como:
%
\begin{equation}
	\Pi_{int,\text{axial}} \ll \Pi_{int,\text{flexión}}
\end{equation}
%
lo que es equivalente a:%
%
\begin{equation}
	\frac{1}{2} \int_{0}^{\ell} E A  \varepsilon_G^2 \dif x \ll \frac{1}{2} \int_{0}^{\ell} EI \left( \frac{\partial^2 v}{\partial x^2}\right)^2 \dif x.
\end{equation}

Usando las expresiones: 
\begin{equation}
	\varepsilon_G(x) = \frac{	N (x) }{E A}
	\quad
	\text{y}
	\quad
	\frac{\partial^2 v}{\partial x^2}(x) = \frac{	M (x) }{E I} ,
\end{equation}
se tiene que esa condición es equivalente a 
%
\begin{equation}
	\int_{0}^{\ell} \frac{	N (x)^2 }{E A} \dif x \ll \int_{0}^{\ell} \frac{	M (x)^2 }{E I}  \dif x..
\end{equation}


Esta condición puede ser asumida al inicio del análisis de un pórtico, y verificada luego de haber calculado los diagramas de directa y momentos flectores. %
%
Se puede decir que para cierto tipo de diagramas, esta condición se cumple si $A \gg I$.

En caso de que esta condición sea asumida, la energía de deformación axial sería despreciada del desarrollo que llevó a la Ecuación~\eqref{eqn:eqrig}, por lo que solo se tendrían los términos de flexión y la barra no tendría desplazamiento relativo, es decir $u_2=u_1$.

En la Sección~\ref{sec:ejenergdef} se presenta un ejemplo de resolución de un pórtico en el cual se calculan las energías de deformación y se verifica la aproximación.


\subsection{Ecuaciones para métodos analíticos}

Los métodos analíticos de resolución se basan en el planteo de las ecuaciones para cada elemento de la estructura independientemente, y luego imponer condiciones de equilibrio de momentos y fuerzas en los nodos de unión correspondientes. %

Para aplicar este método se asume por una parte que la viga no tiene deformación axial (despreciable respecto a la flexional) y por otra parte, se definen variables auxiliares como $\psi$, dada por la expresión:
%
\begin{equation}
  \psi = \frac{v_2 - v_1}{\ell}.
\end{equation}
%

$\psi$ puede ser interpretado geométricamente como el giro antihorario (en pequeños giros) de la recta que une los dos nodos (cuerda). Se puede ver un diagrama en la \autoref{fig:psi}.

\begin{figure}[htb]
	\centering
	\def\svgwidth{0.55\textwidth}
	\input{figs/UT3/diagramaPsi.pdf_tex}
	\caption{Diagrama para interpretación geométrica de $\psi$.}
	\label{fig:psi}
\end{figure}

De esta forma, los desplazamientos nodales de una viga pasan a quedar definidos (a menos de una traslación) por $\theta_1$, $\theta_{2}$ y $\psi$.

Recordando las expresiones deducidas para un elemento de viga en la sección anterior, y aplicando las hipótesis y usando la variable $\psi$ se tiene que las ecuaciones de momentos pasan a estar dadas por: %
%
\begin{equation}\label{eqn:eqvig}
\left\{
\begin{array}{rl}
\displaystyle
\frac{2 EI}{\ell} \left( 2 \theta_1  + \theta_2 - 3 \psi   \right) =& M_1  + M_1^{eq}\\[3mm]
\displaystyle
\frac{2 EI}{\ell} \left( \theta_1 + 2  \theta_2 - 3 \psi  \right) =& M_2 + M_2^{eq}
\end{array}
\right.
\end{equation}


Se definen los momentos de \textit{empotramiento perfecto} $M^{emp}$ como 
$$
\boxed{
M^{emp} = -M^{eq}
}
$$
por lo que las relaciones de la Ecuación~\eqref{eqn:eqvig} pasan a ser entonces:
%
\begin{equation}\label{eqn:ecmomtr}
	\boxed{
		\left\{
		\begin{array}{rl}
			\displaystyle
			\frac{2 EI}{\ell} \left( 2 \theta_1  + \theta_2 - 3 \psi   \right) + M_1^{emp} =& M_1  \\[3mm]
			\displaystyle
			\frac{2 EI}{\ell} \left( \theta_1 + 2  \theta_2 - 3 \psi  \right) + M_2^{emp} =& M_2 
		\end{array}
		\right.
	}
\end{equation}


\cajaconcepto{Empotramiento perfecto}{
Los momentos de empotramiento perfecto de cada tipo de carga en el tramo son los momentos que se obtendría como reacciones al aplicar esa carga en el tramo a una viga bi-empotrada.
}

La expresión de las ecuaciones con los momentos de empotramiento perfecto facilitan la resolución analítica utilizando tablas.

\cajaactividad{
Sea el pórtico mostrado en la figura, donde cada barra tiene sección uniforme de inercia $I$, y está formado por un material con módulo de Young $E$.
\begin{center}
	\def\svgwidth{0.55\textwidth}
	\input{figs/UT3/ejPortSimple.pdf_tex}
\end{center}
\begin{itemize}
	\item Escribir las ecuaciones de relación momento-giros para cada elemento de barra que conforma el pórtico
	\item Encontrar la relación correspondiente a la hipótesis de energía de deformación axial despreciable.
\end{itemize}
}

Por otra parte a partir de combinaciones lineales de las ecuaciones de momento y cortante se obtiene las relaciones:
%
\begin{equation}\label{eqn:eccortr}
\left\{
\begin{array}{rl}
\displaystyle
F_{y,1} + F_{y,1}^{eq} & \displaystyle
= \frac{M_1 + M_2}{\ell} + \frac{M_1^{eq} + M_2^{eq}}{\ell} \\[5mm]
\displaystyle
F_{y,2} + F_{y,2}^{eq} & \displaystyle
= -\frac{M_1 + M_2}{\ell} - \frac{M_1^{eq} + M_2^{eq}}{\ell}
\end{array}
\right.
\end{equation}
%

Agrupando se puede obtener que esto es equivalente a las ecuaciones:
%
\begin{equation}\label{eqn:eccortrB}
\boxed{
\left\{
\begin{array}{rl}
\displaystyle
F_{y,1} & \displaystyle
= \frac{M_1 + M_2}{\ell} + F_{y,1}^{tr-iso} \\[5mm]
\displaystyle
F_{y,2}  & \displaystyle
= -\frac{M_1 + M_2}{\ell} + F_{y,2}^{tr-iso} 
\end{array}
\right.
}
\end{equation}
%
donde
%
\begin{eqnarray}
F_{y,1}^{tr-iso} = - F_{y,1}^{eq} + \frac{M_1^{eq} + M_2^{eq}}{\ell} \nonumber\\ 
F_{y,2}^{tr-iso} = -F_{y,2}^{eq}- \frac{M_1^{eq} + M_2^{eq}}{\ell} \nonumber
\end{eqnarray}
%
donde se puede mostrar que $F_{y,1}^{tr-iso}$ y $F_{y,2}^{tr-iso}$ son las fuerzas nodales correspondientes a las reacciones de una viga simplemente apoyada. %
%
Por ejemplo para una carga puntual aplicada en la posición $x_P$, usando las ecuaciones \eqref{eqn:M1eqP}, \eqref{eqn:M2eqP} y \eqref{eqn:FyeqP}, se tiene:
%
$$
- F_{y,1}^{eq} + \frac{M_1^{eq} + M_2^{eq}}{\ell} = -P \left( 1-\frac{3x_P^2}{\ell^2} + \frac{2x_P^3}{\ell^3} \right) +  P x_P (\ell-x_P)^2 \frac{1}{\ell^3}  -P x_P^2 (\ell-x_P) \frac{1}{\ell^3}
%\end{equation}
%
%Para el otro momento se tiene
%\begin{equation}\label{eqn:M2eqP}
%	M_2^{eq} = -P x_P^2 (\ell-x_P) \frac{1}{\ell^2},
$$

donde simplificando se tiene
$$
- F_{y,1}^{eq} + \frac{M_1^{eq} + M_2^{eq}}{\ell} = -P \left( 1-\frac{x}{\ell} \right)
$$
es decir, la reacción equivalente a una viga simplemente apoyada, con las convenciones de signos consideradas.

%\begin{equation}\label{eqn:M1eqP}
%	M_1^{eq} = P x_P (\ell-x_P)^2 \frac{1}{\ell^2}.
%\end{equation}
%
%Para el otro momento se tiene
%\begin{equation}\label{eqn:M2eqP}
%	M_2^{eq} = -P x_P^2 (\ell-x_P) \frac{1}{\ell^2},
%\end{equation}
%%
%y para las fuerzas equivalentes se tiene
%%
%\begin{equation}\label{eqn:FyeqP}
%	F_{y,1}^{eq} = P \left(1 - \frac{3 x^{2}}{\ell^{2}} + \frac{2 x^{3}}{\ell^{3}}\right) \qquad
%	F_{y,2}^{eq} =
%	P \left(\frac{3 x^{2}}{\ell^{2}} - \frac{2 x^{3}}{\ell^{3}}\right).
%\end{equation}

Esto se desarrollará con mayor detalle en el momento de aplicación en práctico.


\subsection{Expresiones para un extremo articulado}

En el caso de articulaciones en un extremo de la viga se puede obtener una expresión simplificada de la Ecuación~\eqref{eqn:eqvig}. %
%
Se considera una articulación en el nodo 2, entonces se puede imponer que $M_2=0$ obteniendo  
%
\begin{equation}
\frac{2 EI}{\ell} \left( \theta_1 + 2  \theta_2 - 3 \psi  \right) = 0 + M_2^{eq}
\end{equation}
%
por lo que despejando $\theta_{2}$ se tiene
%
\begin{equation} \label{eqn:artictheta2}
\theta_2 = -\frac{\theta_1}{2} + \frac{3}{2} \psi + \frac{M_2^{eq} \ell }{4 EI}.
\end{equation}

Sustituyendo en la ecuación del momento $M_1$ se tiene
%
\begin{equation}
\frac{2 EI}{\ell} \left( 2 \theta_1 + \left( -\frac{\theta_1}{2} + \frac{3}{2} \psi + \frac{M_2^{eq} \ell }{4 EI} \right) - 3 \psi  \right) = M_1 + M_1^{eq}.
\end{equation}
%
Finalmente operando se obtiene:
%
\begin{equation} \label{eqn:ecmomart}
\frac{3 EI}{\ell} \left( \theta_1 - \psi  \right) = M_1 + M_1^{eq} - \frac{ M_2^{eq}}{2}
\end{equation}
%
que puede ser escrita como
%
\begin{equation}
\boxed{
\frac{3 EI}{\ell} \left( \theta_1 - \psi  \right) + M_1^{emp'} = M_1
}
\end{equation}
%
donde $M_1^{emp'} = - M_1^{eq} + \frac{ M_2^{eq}}{2} $ es el momento de empotramiento perfecto (reacción) de una viga empotrada-articulada. %
%
Fácilmente obtenible utilizando tablas.


De forma análoga, si se tiene una articulación en el nodo 1, se impone $M_1=0$ y se tiene
por lo que despejando $\theta_{2}$ se tiene
%
\begin{equation} \label{eqn:artictheta1}
\theta_1 = -\frac{\theta_2}{2} + \frac{3}{2} \psi + \frac{M_1^{eq} \ell }{4 EI}.
\end{equation}

obteniendo la expresión de momento:%
\begin{equation}
\boxed{
	\frac{3 EI}{\ell} \left( \theta_2 - \psi  \right) + M_2^{emp'} = M_2
}
\end{equation}


%\subsection{Viga bi-articulada}
%
%%
%\begin{equation}
%\theta_1 = \frac{L}{3 EI} ( M_1^{eq} - \frac{ M_2^{eq}}{2}) + \psi  
%\end{equation}
%
%sustituyendo en \eqref{eqn:artictheta2} se tiene
%
%\begin{equation}
%\theta_2 = -\frac{L}{6 EI} ( M_1^{eq} - \frac{ M_2^{eq}}{2}) - \psi0.5   + \frac{3}{2} \psi + \frac{M_2^{eq} L }{4 EI}.
%\end{equation}
%

\subsection{Apoyos elásticos}

La consideración de apoyos elásticos en las ecuaciones de equilibrio nodal es equivalente a la vista en la UT2. En este caso se consideran para resortes de desplazamiento como resortes de giro. En ambos casos se adiciona un término de energía de deformación elástica $\Pi_{res}$ tal que  las fuerzas correspondan con:
%
\begin{equation}
F_{res,x} = -k_u u , \quad
F_{res,y} = -k_v v, 
\quad 
M_{res,\theta} = -k_\theta \theta.
\end{equation}


\subsection{Método \textit{Slope-deflection}}

El método \textit{Slope-deflection} (MSD) consiste en la aplicación de las ecuaciones de MD de vigas a preso-flexión como partes de un pórtico, despreciando la energía de deformación axial asociada a la solicitación directa cuando hay flexión. %
%
Esto en la práctica es equivalente a considerar que el elemento tiene deformación $\varep_G$ nula, sin embargo obviamente sigue siendo capaz de transmitir o soportar esfuerzos de directa. %

Para resolver problemas de pórticos es de utilidad aplicar la clasificación de estructura según sus grados de libertad. Se dice que una estructura es \textit{indesplazable} si las incógnitas a determinar por el MSD son únicamente giros nodales, en cambio es \textit{desplazable} si existe al menos una incógnita de desplazamiento. %
%

En práctico se aplicarán metódicamente los pasos del método, el cual puede ser sistematizado de la siguiente forma:
%
\begin{enumerate}
	\item aplicar las hipótesis del método e identificar la mínima cantidad de incógnitas del problema
	%
	\item definir ejes locales en cada barra, para definir el sentido de las fuerzas $F_y$ nodales,
	\item calcular momentos nodales para cada barra: dadas por las Ecuaciones~\eqref{eqn:ecmomtr} para barras con nodos rígidos en ambos extremos y la Ecuación~\eqref{eqn:ecmomart} para barras con un extremo articulado, y la Ecuación~\eqref{eqn:eccortrB} para las fuerzas
	%
	\item calcular fuerzas nodales por barra: dadas por las Ecuaciones~\eqref{eqn:eccortrB}.
	%	$$
	%	EI \left( \dfrac{12}{L^{3}} w_1 + \dfrac{6}{L^{2}} \theta_1 - \dfrac{12}{L^{3}} w_2 + \dfrac{6}{L^{2}} \theta_2 \right)  = V_1 \\
	%$$
	%$$
	%	EI \left( -\dfrac{12}{L^{3}} w_1 - \dfrac{6}{L^{2}} \theta_1 +\dfrac{12}{L^{3}} w_2 - \dfrac{6}{L^{2}} \theta_2 \right) = V_2 \\
	%$$
	%	
	\item plantear ecuaciones de \textbf{equilibrio de momentos} en nodos rígidos. Sea el nodo $i$, conectado a través de vínculo rígido al conjunto de nodos $R_i$, y sea un momento externo aplicado en $i$ $M_i$ también según la convención 2, entonces la ecuación de equilibrio de momento en dicho nodo está dada por:
	$$
	M_i -\sum_{j\in R_i} M_{ij} = 0
	$$
	donde $M_{ij}$ representa el momento nodal en $i$ de la barra $i-j$.
	
	\item plantear ecuaciones de \textbf{equilibrio de pisos}. Para cada \textit{piso} del pórtico se debe verificar el equilibrio de fuerzas horizontales o cortantes. Esto se realiza aislando el elemento o conjunto de elementos horizontales correspondiente de la estructura, considerando las acciones realizadas por las otras barras y planteando el equilibrio de fuerzas.
	
	\item resolver el sistema de ecuaciones
	\item con las incógnitas cinemáticas calculadas, utilizar las ecuaciones de momento y fuerzas nodales para obtener los diagramas de solicitaciones, realizando equilibrios nodales de fuerzas para obtener las directas.
	
\end{enumerate}


\section{Ejemplos}



\subsection{Ejemplo de aplicación del método de Slope-Deflection}

Sea la estructura mostrada en la \autoref{fig:ejem} donde las barras tienen módulo de Young $E$ y sección transversal de Inercia $I$. %
%
Se desea determinar desplazamientos y giros nodales. %
El procedimiento de resolución es presentado de forma esquemática a continuación.

\begin{figure}[htb]
	\centering
\def\svgwidth{0.55\textwidth}
\input{figs/UT3/portic.pdf_tex}
	\caption{Pórtico de ejemplo.}
	\label{fig:ejem}
\end{figure}

La estructura es desplazable y el desplazamiento horizontal de $B$ es una incógnita a determinar. %
%
Se utilizará como variable el desplazamiento del piso $B-C$, dado por $\Delta_{BC} = \psi_{AB} \ell_{AB}$, por lo que de acuerdo con la convención utilizada $\Delta_{BC}$ es positivo para desplazamientos del piso hacia la izquierda.
Las otras variables a determinar son los giros $\theta_B$ y $\theta_C$. %
%
Se utilizarán en todas las ecuaciones las condiciones de contorno: desplazamientos y giros nulos en A y D.


Usando la ecuación de momento en B en la barra BA se tiene:
\begin{equation}
M_{BA} = \frac{2EI}{6} ( 2\theta_B - 3 \frac{ \Delta_{BC} } {6} ).
\end{equation}

Usando la ecuación de momento en B en la barra BC se tiene:
\begin{equation}
M_{BC} = \frac{2EI}{9} (2\theta_B +\theta_C ) + \frac{ P \cdot 3 \cdot 6^2}{9^2}
\end{equation}

Usando el equilibrio de momentos en B ($M_{BA}+M_{BC}=0$) y simplificando se tiene
\begin{equation}
\boxed{
	2EI \left( 2 \left( \frac{1}{6}+\frac{1}{9} \right) \theta_B + \frac{1}{9} \theta_C - \frac{3 }{6^2} \Delta_{BC} \right) = - \frac{4}{3} P.
}
\end{equation}

Esta consiste en la primer ecuación que relaciona las tres incógnitas a determinar.

Por otra parte se realiza el mismo procedimiento para el nodo C. Para la barra BC se tiene
\begin{equation}
M_{CB} = \frac{2EI}{9} (2\theta_C +\theta_B ) - \frac{P \cdot 3^2 \cdot 6}{9^2}
\end{equation}
y para la barra CD
\begin{equation}
M_{CD} = \frac{2EI}{9} (2\theta_C - \frac{3}{9} \Delta_{BC} )
\end{equation}
Usando el equilibrio de momentos en C y simplificando se tiene:
\begin{equation}
\boxed{
	2EI \left( \frac{1}{9} \theta_B + \frac{4}{9} \theta_C - \frac{3 }{9^2} \Delta_{BC} \right) = \frac{2}{3} P.
}
\end{equation}

Finalmente la ecuación de equilibrio de cortantes del piso es equivalente a la ecuación:
\begin{equation}
\frac{M_{AB}+M_{BA}}{\ell_{AB}}
+ 
\frac{M_{CD}+M_{DC}}{\ell_{CD}} = 0
\end{equation}

Calculando la expresiones de los momentos $M_{CD}$ y $M_{AB}$, sustituyendo y simplificando se obtiene:
\begin{equation}
\boxed{
	2EI \left( -\frac{3}{6^2} \theta_B - \frac{3}{9^2} \theta_C + 6 \left( \frac{1}{6^3} +  \frac{1}{9^3} \right) \Delta_{BC} \right) = 0.
}
\end{equation}

El sistema de ecuaciones lineales a resolver es entonces
\begin{equation}
\bfK   
\left[
\begin{matrix}
\theta_B\\
\theta_C\\
\Delta_{BC}
\end{matrix}
\right]
=
P
\left[
\begin{matrix}
-4/3\\
2/3\\
0
\end{matrix}
\right]
\end{equation}
donde $\bfK$ es la matriz del sistema dada por: 
\begin{equation}
\bfK =
2EI \left[
\begin{matrix}
2 \left( \frac{1}{6}+ \frac{1}{9}\right) & \frac{1}{9} & -\frac{3}{6^2}\\
\frac{1}{9} & \frac{4}{9} & -\frac{3}{9^2}\\
-\frac{3}{6^2} & -\frac{3}{9^2} &  6\left( \frac{1}{6^3}  + \frac{1}{9^3}\right)
\end{matrix}
\right]
\end{equation}

Esta matriz es la llamada matriz de rigidez. Previo a la resolución del sistema es importante verificar que la misma es simétrica. %
%
Para resolver se pueden ejecutar los siguientes comandos en Octave:

\begin{verbatim}
K = 2* [ 2*(1/6+1/9)  1/9         -3/6^2 ; ...
          1/9         4/9         -3/9^2 ; ...
         -3/6^2     -3/9^2    6*(1/6^3+1/9^3) ] ;
u = K \ [-4/3; 2/3; 0]
\end{verbatim}


Obteniendo la solución:
\begin{equation}
\theta_B = \frac{ -1.90385 P }{EI}, \theta_C = \frac{ 0.93930
 P }{EI} \text{ y } \Delta_{BC} = \frac{-3.43990 P }{EI}
\end{equation}

Usando Ftool se verifican los resultados. En la \autoref{fig:dia1} se muestran los diagramas de momento y cortante.
\begin{figure}[htb]
	\centering
	\includegraphics[width=0.53\textwidth]{ejUT3}
	\includegraphics[width=0.45\textwidth]{ejUT3cort}
	\caption{Diagramas de momentos (izquierda) y cortante (derecha).}
	\label{fig:dia1}
\end{figure}

En la \autoref{fig:dia2} se muestran los diagramas de deformada y directas.
\begin{figure}[htb]
	\centering
	\includegraphics[width=0.45\textwidth]{ejUT3defor}
	\includegraphics[width=0.53\textwidth]{ejUT3dire}
	\caption{Diagramas de deformada (izquierda) y directas (derecha).}
	\label{fig:dia2}
\end{figure}


\clearpage

%% Energia de Deformación
\subsection{Ejemplo de cálculo de energías de deformación} \label{sec:ejenergdef}

En esta sección se presenta el cálculo de energías de deformación axial y flexional para un pórtico en particular, a modo de confirmación de la hipótesis establecida por el método de \textit{Slope-Deflection}.

Se considera el pórtico mostrado en la \autoref{fig:porticoUdef}, formado por barras de hormigón con $E$=30 GPa, sometido a una carga puntual en el nodo $B$ de valor $P$= 15 kN. Los pilares y la viga tienen sección cuadrada de 0.40 m de lado.

\begin{figure}[htb]
	\centering
	\def\svgwidth{0.6\textwidth}
	\input{figs/UT3/ejemUdef.pdf_tex}
	\caption{Esquema estructural del pórtico.}
	\label{fig:porticoUdef}
\end{figure}

Las incógnitas necesarias para resolver la estructura por el Método de \textit{Slope-Deflection} son $\theta_B$,  $\theta_C$ y $\Delta_{BC}$. Las condiciones a imponer para resolver el problema son:
%
\begin{equation}
M_{BA}+M_{BC} = 0, \quad  M_{CB}+M_{CD} = 0, \quad \text{y} \quad 		F_{BA}+F_{CD} + P = 0.
\end{equation}
%
Planteando las ecuaciones y resolviendo el sistema de ecuaciones, se obtiene:
%
\begin{equation}
	\boxed{
	\theta_B = -4.69\cdot 10^{-4} \text{rad} \qquad \theta_C = -4.69\cdot 10^{-4} \text{rad }\qquad \Delta_{BC} = -1.13\cdot 10^{-2} \text{m}}
\end{equation}

Conocidos los desplazamientos, a partir de las ecuaciones nodales del Método de Slope Deflection se obtienen los valores de momento flector y cortante y se pueden trazar los diagramas de solicitaciones como fue visto en desarrollos anteriores. Los diagramas de Momento y directa son mostrados en la \autoref{fig:ejenergdefdiagramas}.

%%
%Diagrama momento flector\\
%%
\begin{figure}[htb]
	\centering
	\def\svgwidth{0.36\textwidth}
	\input{figs/UT3/porticoUdefM.pdf_tex}
\hfill	
	\def\svgwidth{0.4\textwidth}
   \input{figs/UT3/porticoUdefN.pdf_tex}
	\caption{Diagrama de momento flector en kNm (izq.), diagrama de directa en kN (der.).}
	\label{fig:ejenergdefdiagramas}
\end{figure}


Interesa calcular la energía de deformación interna a flexión $\Pi_{int,\text{flexi\'on}}$ y axial $\Pi_{int,\text{axial}}$ de la estructura, y compararlas. %

La energía de deformación interna axial $\Pi_{axial}$ se calcula elemento a elemento y utilizando que en este caso particular $N(x)$ es uniforme en cada elemento, se tiene,
%
$$
\Pi_{int,\text{axial}} = \sum_{e=1}^{3}\frac{N^2\ell}{2EA}  
$$
%
la cual coincide con la expresión utilizada en el Método de las Fuerzas de la UT2.

La energía de deformación interna de flexión $\Pi_{int,flexi\'on}$ se calcula elemento a elemento mediante tablas de integrales de multiplicación de funciones como las Tablas de Mohr, de donde se tiene, 
%
$$
\Pi_{int,\text{flexión}} = \int_{0}^\ell \frac{M_z(x)^2}{2EI_z}dx.
$$

En la \autoref{tab:Udef} se presenta una comparación elemento a elemento de la energía de deformación interna.

%%%
\begin{table}[htb]
	\centering
	\begin{tabular}{cccc}
		Barra & $\Pi_{int,\text{axial}}$  (kNm) & $\Pi_{int,\text{flexión}}$ (kNm) & $\Pi_{int,\text{axial}} / \Pi_{int,\text{flexión}} \times 100$  (\%) \\ \toprule
		AB             & 3.16E-04                & 3.16E-02                & 1.00                            \\
		BC             & 2.34E-05                & 2.11E-02                & 0.11                            \\
		CD             & 3.16E-04                & 3.16E-02                & 1.00                           
	\end{tabular}
	\caption{Comparación de energías de deformación interna.}
	\label{tab:Udef}
\end{table}
%%%

Se comprueba entonces, que en el caso usual de barras que se encuentren trabajando a directa y momento flector en pórticos planos, es razonable despreciar la energía de deformación interna a directa. %
%
Sin embargo, debe tenerse presente que este no es el comportamiento real de la estructura sino una aproximación y que, en el caso de pilares muy comprimidos, donde el acortamiento elástico puede ser importante, y aún más agravado por el caso de materiales como el hormigón con envejecimiento, esta hipótesis puede comenzar a perder validez. 





\newpage

\section{Ejercicios}
\setcounter{ejercicio}{0}

En cada ejercicio o estructura descrita a continuación, se pide:
%
\begin{enumerate}
  \item Identificar la mínima cantidad de incógnitas a hallar para obtener las solicitaciones de la estructura mediante el método de Slope Deflection.
  \item  Encontrar las ecuaciones que permiten resolver dichas incógnitas.
  \item  Calcular desplazamientos y giros nodales
  \item calcular reacciones y trazar diagramas de solicitaciones.
  \item realizar un diagrama de la deformada
  \item  Modelar las estructuras en un programa computacional y comparar resultados.
\end{enumerate}



\ejercicio

Considere la viga mostrada en la figura, compuesta por madera ($E=10$ GPa) y con una sección transversal rectangular de $20$ cm de ancho y $40$ cm de altura.

\begin{center}
	\def\svgwidth{0.8\textwidth}
	\input{figs/UT3/UT3ej1.pdf_tex}
\end{center}




\ejercicio

La estructura de acero de la figura ($E=210$ GPa), compuesta por una viga (ABC) formada por 1 PNI28 y un pilar (BD) formada por un perfil PNI22.

\begin{center}
	\def\svgwidth{0.8\textwidth}
	\input{figs/UT3/UT3ej2.pdf_tex}
\end{center}




\ejercicio

Considere la estructura de acero ($E=210$ GPa) compuesta por una viga (ABC), dos pilares (AD y BE) y un tensor (BD), cuya sección es 1PNI18. Para el tensor, en este caso, se considera que no hay deformación axial.

\begin{center}
	\includegraphics[width=0.5\linewidth]{UT3ej3}
\end{center}



\ejercicio

En la estructura de hormigón de la figura (E=25 GPa) los pilares son de 20x30 cm mientras que la viga es de 15x40 cm. 

\begin{center}
	\includegraphics[width=0.65\linewidth]{UT3ej4}
\end{center}


\ejercicio

En la estructura de acero de la figura (E=210 GPa), las barras BD, CF y DE están conformadas por 2PNC24 soldados ([]), mientras que la barra AB está compuesta por 1PNI20. 


\begin{center}
	\includegraphics[width=0.6\linewidth]{UT3ej5}
\end{center}



\ejercicio 

En la estructura de hormigón de la figura (E=30 GPa) las secciones son rectangulares de 12x30 cm. 

\begin{center}
	\includegraphics[width=0.5\linewidth]{UT3ej6}
\end{center}


\ejercicio


Sean las siguientes estructuras conformadas por barras de \textbf{EI}=cte y de longitud \textbf{L} sometidas a un desplazamiento impuesto $\Delta$. Se pide calcular los momentos flectores en los empotramientos.


\begin{center}
	\def\svgwidth{0.45\textwidth}
	a) \input{figs/UT3/UT3ejDespa.pdf_tex}
	\def\svgwidth{0.45\textwidth}
	b) \input{figs/UT3/UT3ejDespb.pdf_tex}
\end{center}

\ejercicio

Sea la viga continua que se muestra en la siguiente figura, compuesta por un material con módulo de elasticidad $\bf E$=29 GPa y sección rectangular 20x60 cm$^2$. La viga se encuentra sometida a una carga uniformemente distribuida $q$= 30 kN/m y los apoyos $\bf B$, $\bf C$ y $\bf D$ tienen un asentamiento $\delta_B$=1.5 cm, $\delta_C$=4.0 cm y $\delta_D$=1.9 cm respectivamente. Se pide calcular los giros correspondientes a la mínima cantidad de incógnitas del problema.

\begin{center}
	\includegraphics[width=0.75\linewidth]{UT3ej8}
\end{center}

\ejercicio

Considere el pórtico que se muestra en la siguiente figura, compuesto por barras de secciòn transversal cuadrada de lado 10 cm y mòdulo de Young E= 200 GPa. 
´
\begin{center}
	 \def\svgwidth{0.7\textwidth}
	\input{figs/UT3/UT3ej9.pdf_tex}
\end{center}

Se pide:

\begin{enumerate}
	\item demostrar que la mínima cantidad de incógnitas del problema al aplicar Slope-deflection es: 3. Justificar.
	\item obtener los giros y desplazamientos correspondientes a las incógnitas del problema
\end{enumerate}
 
