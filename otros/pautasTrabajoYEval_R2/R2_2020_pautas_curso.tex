% ----------------------
\documentclass[a4paper,11pt,twocolumn]{article}
%
\usepackage[utf8]{inputenc}
\usepackage[T1]{fontenc}
\usepackage{libertine}
\usepackage{wrapfig}
\usepackage{graphicx}
\usepackage{xcolor}
\usepackage{caption}
\usepackage{hyperref}

\usepackage{framed}
\usepackage{setspace}
\onehalfspacing

\usepackage[spanish]{babel}
\usepackage{hyphenat}

\setlength\parindent{0em}
\setlength\parskip{-0em}

\definecolor{migreen}{rgb}{0.2,0.4,0}%
\definecolor{ugentblauw}{cmyk}{1 0.8 0.3 0.05}%
\definecolor{miblue}{rgb}{0,0.1,0.6}%
\definecolor{bluedark}{rgb}{0,0.1,0.5}%
\colorlet{shadecolor}{ugentblauw}

\setlength{\columnsep}{5mm}
\setlength{\columnseprule}{0.5pt}

\usepackage[left=10mm,right=10mm,top=18mm,bottom=15mm,a4paper]{geometry}

\usepackage{fancyhdr}
\pagestyle{fancy}
\fancyhf{}
\lhead{Pauta de trabajo y evaluación del curso - Edición 2020}
\rhead{Resistencia de materiales 2}
\cfoot{\thepage}

\usepackage{booktabs}

\usepackage{lineno}

\usepackage{titlesec}

\titleformat{\section}
{\setlength\parskip{-0.5em}\normalfont\Large\color{miblue}\bfseries}{\color{miblue}\thesection}{1em}{}[{\color{miblue}\titlerule[2.0pt]}]

\titleformat{\subsection}
{\setlength\parskip{-0.6em}\normalfont\color{migreen}\large\bfseries}{\thesubsection}{1em}{}[{\color{migreen}\titlerule[1.2pt]}]

\titleformat{\subsubsection}
{\setlength\parskip{-1em}\scshape\bfseries\color{bluedark}}{}{1em}{}[{\color{bluedark}\titlerule[0.5pt]}]

\definecolor{light-gray}{gray}{0.95}
\usepackage{tcolorbox}

\newcommand{\cambio}{\color{blue}}


% ---------------------
\begin{document}

\twocolumn[{%
\begin{minipage}[t]{.6\textwidth}%
		\vspace{0mm}%
		\noindent%
		\textbf{Resistencia de Materiales 2 - cod. 1312\\
			Pauta de trabajo y evaluación - 2020}\\
		Versión \today \\
		Instituto de Estructuras y Transporte\\
		Facultad de Ingeniería, Universidad de la República\\
\end{minipage}%
%
\begin{minipage}[t]{.4\textwidth}
  \vspace{-5mm}%
%  \includegraphics[width=.9\textwidth]{logo_udelar} % .21
	%		~\\[1cm]
	% Upper part of the page
%	\includegraphics[width=0.25\textwidth]{logofing}
	\hfill
	\includegraphics[width=0.8\textwidth]{logoudelar}    
\end{minipage} %

\hrulefill

En este documento se establecen las pautas para trabajo y evaluación de la Unidad Curricular Resistencia de Materiales 2 (código 1312). %
%
Se describen los criterios considerados para la realización, entrega y evaluación de los trabajos de laboratorio del curso, así como también, la dinámica a aplicar en las evaluaciones escritas. Estos criterios también aplican a estudiantes recursantes ya que no se revalidan trabajos de laboratorio de ediciones anteriores. %
%
Estas pautas fueron desarrolladas por los docentes del curso con colaboración de docentes de la Unidad de Enseñanza de FIng.
\vspace{5mm} }]

\section{Modalidad de dictado y evaluación}

\subsection{Vías de comunicación}
	
Todas las comunicaciones de estudiantes a docentes, y viceversa, serán realizadas a través de herramientas institucionales (siempre que los recursos lo permitan) con un objetivo establecido. Los estudiantes contarán con foros específicos para el armado de grupos, planteo de dudas generales, planteo de dudas de práctico, presentación de trabajos entre otros. Los docentes cuentan con foros de novedades, recursos de EVA y encuentros vía Zoom para cumplir con objetivos de acompañamiento de los aprendizajes.

Cualquier planteo o consulta que no se corresponda con las herramientas disponibles en EVA (como por ejemplo lo indicado en la Sección~\ref{sec:justif}), podrá, deberá y se agradecerá sea realizado por correo escribiendo al \textbf{responsable del curso} a: \textit{\href{mailto:jorgepz@fing.edu.uy}{jorgepz@fing.edu.uy}}.

\subsection{Modalidad de dictado de clases}

Durante el curso se subirán contenidos (videos, documentos pdf, encuestas, etc)  para acompañar el proceso de aprendizaje tanto de teórico como de práctico.

El acompañamiento del aprendizaje de teórico será a través 
de dos clases por semana de una hora y media cada una apoyadas en el libro de texto y eventualmente subida de videos complementarios. %
%
El acompañamiento de práctico será realizado a través de la resolución de los ejercicios del libro, contenidos como documentos o videos subidos a EVA y cada semana se tendrán dos horarios de consulta vía zoom (o foros) en horarios publicados.

\subsection{Forma de evaluación}

La evaluación del curso se realiza a través de laboratorios y parciales, totalizando 100 puntos. La aprobación de curso se obtiene con 25 puntos y la exoneración del examen con 60 puntos. %
%
La distribución de puntos se construye como:
%
\begin{description}
\item [Laboratorio] 15 puntos
%
\item [Simulacro de primer parcial] 10 puntos: prueba escrita de práctico  a realizarse en modalidad virtual el día jueves 8 de octubre.
%
\item [Primer parcial] 25 puntos: prueba escrita virtual con ejercicios prácticos y misma modalidad que en el simulacro.
%
\item [Segundo parcial] 50 puntos: prueba escrita \textbf{presencial} con parte teórico y parte práctica incluyendo todos los contenidos del curso.
{\color{red}
AGUARDANDO CONFIRMACIÓN DE AUTORIDADES}
\end{description}


\subsection{Justificaciones previas al parcial} \label{sec:justif}

En el caso de que cualquier estudiante tenga cualquier tipo de dificultad para realizar la prueba (problemas de conexión, viajes, etc) deberá comunicarlo al \textbf{responsable del curso} con una antelación mínima de \textbf{diez días hábiles}, para que los docentes estén al tanto. En el caso de dificultades de aprendizaje o expresión se agradecerá la presentación de una nota firmada por un profesional especificando el tipo de dificultad y recomendaciones sobre las consideraciones a tener con el/la estudiante por parte de los docentes.

\section{Laboratorios}

\subsection{Aspectos generales}

\textbf{Formación de grupos}
Los trabajos serán realizados en \textbf{grupos de 4} estudiantes. Los estudiantes deberán formar y comunicar los grupos a través de la actividad Elección de grupos en EVA \textbf{antes del día viernes 11/9 a las 23:55 hrs}. %
%
También deberán subir a través de un formulario en eva un documento pdf de 1 página indicando la estructura seleccionada con fotos descriptivas. En caso de constatarse similitudes con trabajos de años anteriores u otras estructuras presentadas por otros grupos en el mismo año, se deberá modificar la estructura. La hora de subida del archivo será considerada como criterio para definir qué grupo debe modificar su estructura.

\vspace{5mm}

\textbf{Sobre el problema a resolver}
Cada grupo considerará una estructura existente para realizar diferentes análisis aplicando los conceptos vistos en el curso. %
%
La estructura debe tener una complejidad tal que pueda ser analizada considerando esquemas básicos de cálculo (EBC) planos y tridimensionales. Las estructuras planas con carga perpendicular al plano de la misma serán consideradas como tridimensionales. %
%
Se modelarán utilizando elementos de barra, viga o pórtico. %
En caso de que existan elementos planos, como losas, se deberá utilizar algún criterio para la distribución de las cargas correspondientes a los elementos de viga vinculados.
%
Se podrán considerar análisis de posibles modificaciones a realizar sobre estructuras existentes. %
%
En el caso de que se consideren estructuras isostáticas se podrán proponer modificaciones para obtener una estructura hiperestática. Se podrán considerar excepciones a estas condiciones en coordinación con los tutores asignados.

\vspace{5mm}

\textbf{Ejemplos} de tipos de estructuras que se podrán considerar:
%
\begin{itemize}
	\item \textbf{puentes}: puentes viga, puentes atirantados, puentes losa y puentes en arco,
	\item \textbf{marítimas}: muelles, estructuras flotantes , diques, canales de navegación,	
	\item \textbf{torres reticuladas}: mástiles atirantados, torres de tendido eléctrico, pórticos de recepción,
	\item \textbf{edificaciones}: casas, edificios, estadios de fútbol,
	\item \textbf{industriales}: puente grúa, grúa de carga portuaria,
	\item \textbf{otras estructuras}: alumbrado público, cartelería, señalización, etc.
\end{itemize}

A modo de guía y apoyo están disponibles los laboratorios de años anteriores. Las estructuras seleccionadas para el año corriente \textbf{deberán ser distintas} o presentar variantes respecto a las elegidas en trabajos anteriores.	

\subsection{Entregas de informe}

La entrega y evaluación del trabajo será a través de entregas parciales de acuerdo a las pautas descritas a continuación. %
%
Cada entrega parcial es parte de un trabajo global acumulativo. Cada entrega es autocontenida, es decir que no se deben realizar referencias a entregas anteriores. %
%
Tanto la descripción de los puntos solicitados como las fechas de entrega podrán tener ligeras variaciones/aclaraciones durante el transcurso del semestre.

\subsubsection{Entrega 1}
\underline{Entrega inicial.} %
%
Un integrante de cada grupo deberá subir en el espacio habilitado en el EVA \textbf{únicamente un documento pdf de 4 páginas como máximo} en el cual se incluyan los resultados de los siguientes puntos:
\begin{itemize}
  \item escoger una estructura existente y presentar fotos que permitan apreciar la geometría de la misma.
  \item clasificar la estructura según criterios vistos en clase.
  \item establecer hipótesis sobre el comportamiento constitutivo y propiedades de los materiales que forman la estructura.
  \item enumerar al menos dos estados de cargas a la cual la estructura puede estar sometida durante su vida útil.
  \item presentar dos esquemas de cálculo simplificados de estructuras planas. Definir estados de carga, geometría de las barras (longitud y sección transversal) y vínculos entre elementos. %
	%
	Al menos uno de los esquemas debe permitir resolución analítica.
	\item calcular desplazamientos nodales y presentar la deformada para ambos esquemas (para el esquema de solución analítica determinar analíticamente el desplazamiento de algunos puntos de referencia).
		%
	\item validar los resultados analíticos comparando con resultados numéricos (herramientas computacionales).
	
\end{itemize}	

El trabajo deberá ser enviado antes del día \textbf{viernes 23/10 a las 23:55 hrs}.
%

En caso de constatarse similitudes con trabajos de años anteriores la entrega será rechazada. La misma deberá ser realizada nuevamente, considerándose un factor de penalización de 0,5 para el puntaje total de la entrega.
%

De existir coincidencias entre grupos cursantes, la entrega será aceptada, corregida y puntuada. Sin embargo se realizará un sorteo para determinar cuál de los grupos involucrados deberá modificar la estructura seleccionada para las siguientes instancias del laboratorio.

\subsubsection{Entrega 2}

\underline{Entrega de análisis analíticos y modelos computacionales.}
Se deberán entregar en un archivo \textbf{zip}, tanto el documento del informe del trabajo como los archivos de los modelos/códigos utilizados para los análisis. %
%
El informe tendrá \textbf{como máximo 6 páginas} y los puntos a cubrir son:
%
\begin{itemize}
	%
	\item presentar diagramas de directa, momento y cortante para los dos esquemas de cálculo simplificados de estructuras planas.
	%
	\item comparar los resultados de desplazamientos y solicitaciones obtenidos para los diferentes esquemas de cálculo tomando puntos y elementos de referencia.
    \item  presentar y analizar al menos un esquema básico de cálculo de estructura tridimensional de barras.
	\item presentar diagramas de solicitaciones (4 solicitaciones mas relevantes) y deformada. Se podrá utilizar herramientas computacionales.
	\item comparar los resultados obtenidos con los de los modelos simplificados planos. Calcular errores y diferencias relativas
    \item el informe debe contener toda la información necesaria para comprender el trabajo realizado, incluyendo los puntos más importantes de la entrega 1 (alguna foto de la estructura, descripción de hipótesis sobre los materiales, etc.).
\end{itemize}	

Esta entrega deberá ser enviada \textbf{antes del día lunes 28/10 a las 07:55 hrs.} %
%

\subsubsection{Entrega Final}

Se deberá subir el documento actualizado con todas las correcciones y/o comentarios recibidos luego de la presentación oral. Esta entrega será la versión a publicar al año siguiente para los nuevos estudiantes del curso y no tendrá puntaje asignado.  
%
Esta entrega deberá ser subida al EVA \textbf{antes del día miércoles 11/11 a las 23:55 hrs}. %

\subsection{Exposición oral}

Todos los grupos realizarán una presentación oral \textbf{virtual }del trabajo \textbf{transmitida de forma abierta }a sus compañeros, docentes e invitados \textbf{el día sábado 21/11}. %
%
Esta instancia de evaluación es obligatoria para todos los integrantes de los grupos y el orden de presentación será sorteado el mismo día.
%
Dos estudiantes del grupo elegidos por el tribunal en el momento de la presentación harán la exposición oral. Se podrán formular preguntas breves sobre el trabajo o conceptos vistos en el curso a cualquier integrante del grupo.

Las presentaciones deberán ser entregadas en formato \textbf{pdf} antes de la fecha a ser definida. %
%
El documento utilizado para la ppt puede ser realizado usando cualquier herramienta (Libreoffice, PowerPoint, Beamer-\LaTeX, etc), pero \textbf{debe ser subido en formado pdf}. %

\subsection{Licencia de publicación}

Los trabajos podrán ser publicados en repositorios de la Universidad de la República bajo una licencia \textit{Creative Commons Attribution-ShareAlike 4.0 International License}. Ver detalles en \href{https://creativecommons.org/licenses/by-sa/4.0/}{creativecommons.org/licenses/by-sa/4.0}.

Los autores del trabajo serán los integrantes del grupo, siguiendo el mismo orden que los estudiantes elijan y utilicen en la primer página del informe.
%
\subsection{Criterios de asignación de puntaje}

El puntaje final obtenido del laboratorio \textbf{PL}, con un valor máximo de \textbf{15 puntos}, será calculado usando la siguiente ecuación:
\begin{eqnarray}
\textbf{PL} &=& 0.3 \, \textbf{PE1} + 0.5 \, \textbf{PE2} + 0.2 \, \textbf{PExp} \nonumber
\end{eqnarray}

donde \textbf{PE1} y \textbf{PE2} son los puntajes obtenidos en las dos primeras entregas y \textbf{PExp} es el puntaje de la defensa/exposición oral del trabajo. \textbf{Los grupos que no se presenten a la exposición oral tendrán puntaje 0} en el puntaje final  del laboratorio. %
%
Además, en caso de constatarse falta de colaboración en alguno de los integrantes del grupo, \textbf{ se podrán realizar penalizaciones adicionales sobre el estudiante.} %


\subsubsection{Puntaje de informes}

Se asignarán puntos en una escala de 0 a 3 para cada uno de los siguientes criterios:
%
\begin{enumerate}
	\item Cumplimiento de la consigna
	\item Figuras y tablas
	\item Redacción
	\item Referencias bibliográficas y otros
	\item Formato
\end{enumerate}
%
Los puntajes obtenidos en cada criterio serán sumados considerando factores de ponderación y se obtendrá un valor entre 0 y 15. %
%
\textbf{Las entregas fuera de plazo} serán corregidas para darle una devolución al estudiante pero \textbf{tendrán puntaje 0}. %

El estilo a utilizar es el definido en el \textit{template} de \LaTeX \, disponible en el sitio eva. %
%
Se podrá utilizar cualquier editor aunque se deberá entregar en formato \textbf{pdf} y respetar el estilo definido: fuente: Times new roman; tamaño texto normal: 11; diagramación de página doble columna; numeración de secciones; formato de referencias, márgenes 1.5 cm en todos los bordes excepto 2 cm en margen superior, encabezado y pié de página, tamaño de títulos de tablas y figuras, etc.


\paragraph{1) Cumplimiento de la consigna} %
%
\begin{itemize}
	\item \textbf{3}: el trabajo cumple con la consigna de la entrega, presenta claramente todos los puntos pedidos y no presenta incoherencias en los resultados.
	\item \textbf{2}: el trabajo cumple correctamente con la mayoría de los puntos solicitados aunque algunos puntos están incompletos o presentan resultados con algunas incoherencias (mencionados por los estudiantes). %
	%
	Ejemplo: se presenta algún diagrama de magnitudes relevantes, como por ejemplo solicitaciones, y no es aclarado, ni en la figura ni en el texto, en qué unidades están expresados lo valores.
	%
	\item \textbf{1}: el trabajo cumple correctamente con parte de los puntos solicitados aunque varios puntos están incompletos y/o presentan resultados incoherentes y esto no es mencionado.
	%
	\item \textbf{0}: el trabajo no cumple correctamente ninguno de los puntos solicitados.
\end{itemize}

\paragraph{2) Figuras y tablas}
\begin{itemize}
	\item \textbf{3}: se utiliza una cantidad adecuada de figuras y tablas, de forma ordenada e integrada con el texto. En los casos que requiera (como diagramas) permiten al lector leer números o ver detalles, en los casos en los que no es posible se aclara en el texto.
	\item \textbf{2}: en algunos casos las figuras no son legibles o algunas tablas tienen errores menores,
	\item \textbf{1}: las figuras o tablas no están integradas con el texto por no ser referenciadas en el mismo o se usan figuras o tablas con errores,
	\item \textbf{0}: la mayoría de las figuras utilizadas no tienen información relevante o no son claras.
\end{itemize}


\paragraph{3) Redacción}
\begin{itemize}
	\item \textbf{3}: el texto es claro y conciso, no se comenten errores ortográficos o gramaticales en cantidad considerable,
	\item \textbf{2}: se encuentran errores gramaticales menores y la redacción es suficientemente clara,
	\item \textbf{1}: se encuentra una cantidad importante de errores ortográficos, especialmente tildes, y/o la redacción no es suficientemente clara,
	\item \textbf{0}: el texto no es claro, tiene: oraciones incompletas, mal uso de puntuación o mayúsculas.
\end{itemize}

\paragraph{4) Referencias bibliográficas y otros}
\begin{itemize}
	\item \textbf{3}: se citan referencias bibliográficas de forma adecuada así como también se definen claramente las fuentes de otros materiales usados como imágenes o datos obtenidos de sitios web,
	\item \textbf{2}: existen algunas referencias puntuales faltantes,
	\item \textbf{0}: no se hace un uso correcto de referencias.
\end{itemize}

\paragraph{5) Formato}
\begin{itemize}
	\item \textbf{3}: se cumplió con el estilo definido,
	\item \textbf{2}: se cumplió con el estilo definido pero no se respetó el máximo de páginas,
	\item \textbf{0}: no se cumplió con el estilo definido.
\end{itemize}



\subsubsection{Puntaje de exposición}


\begin{itemize}
	\item \textbf{Exposición oral:} %
	%
	\begin{itemize}
		\item \textbf{3}: ambos estudiantes realizan la presentación de forma respetuosa y correcta dirigiéndose al público y en el tiempo asignado, el contenido de la presentación es concreto y orientado a mostrar los resultados más importantes del trabajo,
		\item \textbf{2}: la presentación se realiza correctamente hasta que el tiempo máximo es alcanzado sin poder finalizar la misma o se cometen errores menores de conexión con el público y visualización de la presentación,
		\item \textbf{1}: se presenta en tiempo adecuado pero se cometen errores al presentar sin lograr exponer claramente conceptos de la presentación,
		\item \textbf{0}: se realiza una presentación incompleta con mal uso del tiempo y cometiendo los errores mencionados anteriormente.
	\end{itemize}
	
	\item \textbf{Preguntas}:
	\begin{itemize}
		\item \textbf{3}: ambos estudiantes responden correctamente,
		\item \textbf{2}: se comenten errores de importancia menor,
		\item \textbf{0}: se cometen errores importantes.
	\end{itemize}
	
\end{itemize}
%

Los puntajes obtenidos en cada criterio serán sumados considerando factores de ponderación y se obtendrá un valor \textbf{PExp} comprendido entre 0 y 15. %

\textbf{Las presentaciones deberán ser realizadas en un tiempo máximo de 9 minutos.} Luego se tomarán algunos minutos para realizar preguntas.



\section{Parciales}

La dinámica de las pruebas parciales será publicada luego de la confirmación por partes de las autoridades de Facultad, teniendo en cuenta la evolución de la situación sanitaria.


\subsection{Simulacro y primer parcial}

El simulacro y el primer parcial tendrán la misma modalidad. El simulacro será realizado el jueves 8 de Octubre de 8 a 10 hrs de la mañana. El día y horario del primer parcial será anunciado cuando haya una fecha definida.

Está previsto que la modalidad de ambas pruebas cuenta con las siguientes características:
\begin{itemize}
\item Habrá una tarea de EVA para subir el desarrollo. El desarrollo deberá ser subido en un \textbf{único archivo en formato pdf} para que sea posible corregir en el propio archivo y facilitar la propia muestra del parcial luego.
%
\item Habrá un cuestionario de EVA para responder todas las preguntas, ingresando valores numéricos, permitiendo al estudiante cambiar de página hacia adelante o atrás durante todo el período de la prueba.
%
\item Habrá un salón de clase de zoom habilitado durante toda la prueba para responder consultas de letra o hacer aclaraciones generales.
%
\item El contenido de esta prueba será práctico, evaluando los aprendizajes asociados a las Unidades Temáticas 2 y 3. Habrá un ejercicio por cada uno de estos temas.
%
\item En la hora fijada de inicio para cada evaluación, se habilitará una carpeta en EVA de donde deberán descargar la letra del parcial. Habrá una letra en formato pdf para cada estudiante, debiendo cada uno descargar la que le corresponda.
\end{itemize}

% ---------- PREPANDEMIA ------------------ 
%\paragraph{Instancias de la prueba}
%La prueba será tomada en el día definido a través de dos instancias, separadas por un intervalo de 15 minutos para descanso fuera del salón. %
%%
%\begin{enumerate}
%\item La primera instancia tendrá una duración de 1 hora y estará orientada a evaluar mayoritariamente conocimientos teóricos. %
%%
%\item La segunda instancia tendrá una duración de 2 horas y se evaluarán habilidades relacionadas principalmente a la resolución de problemas prácticos. %
%%
%\end{enumerate}
%
%\paragraph{Materiales}
%Durante la instancia práctica se podrá utilizar todo tipo de material. %
%%
%Para la instancia teórica los estudiantes podrán utilizar únicamente materiales para escribir. Las hojas serán suministradas por los docentes y no será necesario utilizar calculadora. %
%%
%
%\paragraph{Dinámica de la prueba}
%A la hora y lugar establecidos se llamará a los estudiantes para ingresar al salón para realizar la prueba teórica. %
%%
%Luego de finalizada la prueba teórica los estudiantes se retirarán del salón. %
%%
%Luego de finalizado el tiempo máximo definido para la prueba teórico los estudiantes saldrán del salón durante 15 minutos para ser llamados luego para la instancia práctica.
%
%	\paragraph{Puntaje}
%La primer instancia parcial otorgará un puntaje máximo de 35 puntos, mientras que la segunda instancia parcial otorgará un puntaje máximo de 50 puntos. %
%%
%
%	\paragraph{Resultados}
%Los resultados serán publicados en un plazo no mayor a 15 días hábiles. %
%%
%Junto con los resultados serán enumerados los errores más frecuentes cometidos por los estudiantes.	 %
%

%\section{Exámenes}
%
%El formato de examen será definido por el tribunal correspondiente. %
%%
%A pesar de esto, se describen de forma sintética los lineamientos de la dinámica que se considera aplicar a partir de esta edición del curso. %
%%
%
%El examen tiene una parte escrita y una parte oral. Aquellos estudiantes que alcancen un cierto puntaje mínimo en la parte escrita pasarán a la parte oral. %
%%
%En la parte oral deberán responder dos preguntas. Las respuestas dadas por los estudiantes serán ponderados con el puntaje obtenido en la parte escrita de acuerdo con una ecuación previamente establecida por el tribunal.

% ----------------------------------------------------


\end{document}
