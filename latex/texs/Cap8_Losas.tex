% libroResMat2
% Copyright (C) 2020  J.M. Perez Zerpa, et. al.
%
% This program is free software: you can redistribute it and/or modify
% it under the terms of the GNU General Public License as published by
% the Free Software Foundation version 3 of the License.
%
% This program is distributed in the hope that it will be useful,
% but WITHOUT ANY WARRANTY; without even the implied warranty of
% MERCHANTABILITY or FITNESS FOR A PARTICULAR PURPOSE. See the
% GNU General Public License for more details.
%
% You should have received a copy of the GNU General Public License
% along with this program.  If not, see <http://www.gnu.org/licenses/>.

\chapter{Introducción al análisis de losas}


En esta unidad temática (desarrollada en clase) se presentan conceptos básicos e introductorios del análisis de losas. Se presentan las líneas principales de la teoría tomando \cite{Onate2013} como referencia. 
%
El procedimiento para obtener soluciones analíticas es descrito en la referencia \citep{Reddy2002b}.
%
Respecto a aplicaciones e intepretación de resultados se presenta el código \ref{cod:femplateexample}, utilizandolo como herramienta para introducir los diferentes tipos de condiciones de contorno.