% libroResMat2
% Copyright (C) 2020  J.M. Perez Zerpa, et. al.
%
% This program is free software: you can redistribute it and/or modify
% it under the terms of the GNU General Public License as published by
% the Free Software Foundation version 3 of the License.
%
% This program is distributed in the hope that it will be useful,
% but WITHOUT ANY WARRANTY; without even the implied warranty of
% MERCHANTABILITY or FITNESS FOR A PARTICULAR PURPOSE. See the
% GNU General Public License for more details.
%
% You should have received a copy of the GNU General Public License
% along with this program.  If not, see <http://www.gnu.org/licenses/>.

\chapter{Soluciones de los ejercicios}

\section{Ejercicios de la UT1}

\begin{description}
\item [1.1] 
\begin{enumerate}[label=\alph*)]
\item Diagrama de cortantes en kN

\begin{center}
	\includegraphics[width=.85\textwidth]{ej11a}
\end{center}

Diagrama de momentos en kNm

\begin{center}
	\includegraphics[width=.85\textwidth]{ej11b}
\end{center}

\item  $b = 15$ cm.
\item
\begin{itemize}
	\item $\theta_A = 6.32 \times 10^{-3}$ rad  horario,
	\item $\theta_B = 4.74 \times 10^{-3}$ rad antihorario,
	\item $\delta_C = 3.95 \times 10^{-3}$ m hacia arriba.
\end{itemize}

\end{enumerate}

\item[1.2] 

\begin{enumerate}[label=\alph*)]
	\item Directas en kN


\begin{center}
	\includegraphics[width=.85\textwidth]{12a}
\end{center}

cortantes en kN

\begin{center}
	\includegraphics[width=.85\textwidth]{12b}
\end{center}

momentos en kN m

\begin{center}
	\includegraphics[width=.85\textwidth]{12c}
\end{center}
\addtocounter{enumi}{1}

\item 
\begin{itemize}
\item $\theta_D = (53.33$ kN m$^2)$/EI horario.
\item $ \theta_G = (13.33$ kN m$^2)$/EI antihorario.
\item $ \delta_C = (53.33 $ kN m$^3)$/EI hacia arriba y $(80$ kN m$^3)$/EI hacia la izquierda.
\item  $\delta_H = (26.67$ kN m$^3)$/EI hacia abajo y $(80$ kN m$^3)$/EI hacia la izquierda.
\end{itemize} 
\end{enumerate}

%
\item[1.3]

\begin{enumerate}[label=\alph*)]
	
\item $P = +210$ kN / $- 163.3$ kN.

\item $2$ PNC $6.5$.

\end{enumerate}
	%
\item [1.4]


\begin{enumerate}[label=\alph*)]
	
\item
	
	\begin{center}
		\includegraphics[width=.85\textwidth]{14a}
	\end{center}
\item 
\begin{itemize}
\item RC
\begin{itemize}
	\item Máximo: De A a D $\rightarrow$ RC,MAX = 120 kN.
    \item Mínimo: Sin carga $\rightarrow$ RC,MIN = 0 kN.
    \end{itemize}
\item VP
\begin{itemize}
	\item Máximo: De A a C $\rightarrow$ VC,MAX= 20 kN.
    \item Mínimo: De C a P $\rightarrow$ VC,MIN = -80 kN.
    \end{itemize}
\item MS
\begin{itemize}
	\item Máximo: De A a C y de E a F $\rightarrow$ MC,MAX = 50 kN m.
   \item Mínimo: De C a E $\rightarrow$ MC,MIN = -160 kN m.
\end{itemize}
\end{itemize}

\end{enumerate}
\item [1.5]
%
Reacciones,
\begin{itemize}
\item MA = $-5.2$ kN m.
\item  RA = $5.04$ kN.
\item RB $= 24.96$ kN + 40 kN  $= 64.96$ kN.
\end{itemize}
%


Cortante en kN

\begin{center}
	\includegraphics[width=.65\textwidth]{15a}
\end{center}

Momentos en kN m
 
 \begin{center}
 	\includegraphics[width=.65\textwidth]{15b}
 \end{center}

\end{description}
%
\section{Ejercicios de la UT2}
%
\begin{description}
\item [2.1]
%
Reacciones
\begin{itemize}
\item RB = 50 kN hacia la izquierda y 50 kN hacia abajo.
\item RC = 50 kN hacia arriba.
\end{itemize}


Directas en barras
\begin{center}
\begin{tabular}{lr}
\hline
Barra & N [kN] \\
\hline
AB & 50 \\
BC & 0 \\
AC & $-70.7$\\
\hline
\end{tabular}
\end{center}

Desplazamiento de A: $2.44 \times  10^{-3}$ m hacia la derecha y $6.38 \times 10^{.4}$ m hacia arriba.

%
\item [2.2]

Reacciones
\begin{itemize}
\item RA = 13,48 kN hacia abajo (reacción horizontal en A nula).
\item RB = 33,04 kN hacia abajo.
\item  RC = 106,52 kN hacia arriba.
\end{itemize}

Directas en barras
\begin{center}
	\begin{tabular}{lr}
		\hline
Barra & N [kN] \\
\hline
AB & $-18.0$ \\
AE & $22.5$ \\
BE & $46.5$ \\
BC & $-22.5$ \\
EC & $-62.0$ \\
EF & $100.0$ \\
CF & $-60.0$ \\
CD & $-100.0$ \\
FD & $80.0$ \\
\hline
\end{tabular}
\end{center}

Desplazamiento de D: $ 2.15 \times 10^{-3}$ m hacia la derecha y $4.8 \times 10^{-3}$ m hacia abajo.

\item[2.3]

Reacciones
\begin{itemize}
\item RA = 20 kN hacia la izquierda.
\item RD = 20 kN hacia la derecha y 10 kN hacia arriba.
\end{itemize}

\begin{center}
	\begin{tabular}{lr}
		\hline
		Barra & N [kN] \\
		\hline
AB & $15.68$ \\
BC & $4.42$ \\
AD & $-4.42$ \\
BE & $0.00$ \\
CF & $4.42$ \\
AE & $6.25$ \\
DB & $-7.89$ \\
BF & $7.89$ \\
CE & $-6.25$ \\
DE & $-14.42$ \\
EF & $-5.58$ \\
\hline
\end{tabular}
\end{center}

Desplazamiento de F: $2.38 \times 10^{-4}$ m hacia la izquierda y $9.84 \times 10^{-4}$ m hacia abajo.

\item[2.4]

Reacciones
\begin{itemize}
\item RA = $5.24$ kN (reacción horizontal en A nula).
\item RF = $9.52$ kN.
\item RG = $5.24$ kN.
\end{itemize}

Directas en barras

\begin{center}
	\begin{tabular}{lr}
		\hline
		Barra & N [kN] \\
		\hline
AB & $-7.86$ \\
BC & $-7.86$ \\
AD & $9.45$ \\
DE & $0.72$ \\
EC & $9.45$ \\
BD & $8.58$ \\
BE & $8.58$ \\
FB & $9.52$ \\
GC & $5.24$ \\
\hline
\end{tabular}
\end{center}

Desplazamiento vertical de B: $5.44 \times 10^{-3}$ m hacia abajo. Desplazamiento vertical de C $ 2.99 \times 10^{-3}$ m hacia abajo

\item[2.5]

\begin{enumerate}[label=\alph*)]
	\item
Directas en barras

\begin{center}
	\begin{tabular}{lr}
		\hline
		Barra & N [kN] \\
		\hline
AB & $0.086$ P \\
AF & $-0.172$ P \\
BC & $0.259$ P \\
BF & $0.172$ P \\
BG & $-0.172$ P \\
CD & $0.313$ P \\
CG & $-0.982$ P \\
CH & $-0.928$ P \\
CI & $-0.094$ P \\
DE & $0.086$ P \\
DH & $-0.226$ P \\
DI & $0.226$ P \\
EI & $-0.172$ P \\
FG & $-0.172$ P \\
GH & $0.233$ P \\
HI & $-0.118$ P \\
\hline
\end{tabular}
\end{center}

\item PNI 16
\end{enumerate}








\section{Ejercicios de la UT3}

\item[3.1] 
Mínima cantidad de incógnitas: 1, $\theta_B = -1.94 \times 10^{-3}$ rad.

Diagrama de cortantes (kN)

	\begin{center}
	\includegraphics[width=.85\textwidth]{ej31V}
\end{center}

Diagrama de momentos (kN.m)

	\begin{center}
	\includegraphics[width=.85\textwidth]{ej31M}
\end{center}

\item[3.2]
Mínima cantidad de incógnitas: 1 $\theta_B = 3.60 \times 10^{-3}$ rad.

Diagrama de directas (kN)

	\begin{center}
	\includegraphics[width=.85\textwidth]{ej32N}
\end{center}

Diagrama de cortantes (kN)

	\begin{center}
	\includegraphics[width=.85\textwidth]{ej32V}
\end{center}

Diagrama de momentos (kN.m)

	\begin{center}
	\includegraphics[width=.85\textwidth]{ej32M}
\end{center}

\item [3.3]

Mínima cantidad de incógnitas: 2, $\theta_A = -2.70 \times 10^{-3}$ rad y $\theta_B = 2.69 \times 10^{-3}$ rad.

$\mcN$ (kN)

	\begin{center}
	\includegraphics[width=.45\textwidth]{ej33N}
\end{center}

$\mcV$ (kN)

	\begin{center}
	\includegraphics[width=.45\textwidth]{ej33V}
\end{center}

$\mcM$ (kNm)

	\begin{center}
	\includegraphics[width=.45\textwidth]{ej33M}
\end{center}

\item[3.4]
Mínima cantidad de incógnitas: 3, $\theta_B = -1.66 \times 10^{-3}$ rad, $\theta_C = 1.37 \times 10^{-3}$ rad y $\Delta = 6.75 \times 10^{-3}$ m (hacia la derecha).


$\mcN$ (kN)

\begin{center}
	\includegraphics[width=.85\textwidth]{ej34N}
\end{center}

$\mcV$ (kN)

\begin{center}
	\includegraphics[width=.85\textwidth]{ej34V}
\end{center}

$\mcM$ (kNm)

\begin{center}
	\includegraphics[width=.85\textwidth]{ej34M}
\end{center}


\item [3.5]

Mínima cantidad de incógnitas: 3, $\theta_C = -2.99 \times 10^{-3}$ rad, $ \theta_D = 3.04 \times 10^{-3}$ rad y $\Delta  = 14.42 \times 10^{-3}$ m (hacia la derecha).
%

$\mcN$ (kN)

\begin{center}
	\includegraphics[width=.45\textwidth]{ej35N}
\end{center}

$\mcV$ (kN)

\begin{center}
	\includegraphics[width=.85\textwidth]{ej35V}
\end{center}

$\mcM$ (kNm)

\begin{center}
	\includegraphics[width=.85\textwidth]{ej35M}
\end{center}



\item[3.6]
Mínima cantidad de incógnitas: 3, $\theta_B = 8.57 \times 10^{-4}$ rad, $\theta_C = 3.91 \times 10^{-4}$ rad y $\Delta  = 4.62 \times 10^{-3} $ m (hacia abajo).

$\mcN$ (kN)

\begin{center}
	\includegraphics[width=.45\textwidth]{ej36N}
\end{center}

$\mcV$ (kN)

\begin{center}
	\includegraphics[width=.45\textwidth]{ej36V}
\end{center}

$\mcM$ (kNm)

\begin{center}
	\includegraphics[width=.45\textwidth]{ej36M}
\end{center}

\item[3.7]

$a)$

\begin{center}
	\includegraphics[width=.45\textwidth]{3.7a}
\end{center}

$b)$

\begin{center}
	\includegraphics[width=.45\textwidth]{3.7b}
\end{center}


\section{Ejercicios de la UT4}

Las soluciones de los ejercicios están planteadas en base a la hipótesis de indeformabilidad de directa de aquellas barras que presentan flexión. Dicha hipótesis deberá ser corroborada para los distintos casos.

\item[4.1] 

$\mcN$ (kN)
\begin{center}
	\includegraphics[width=.95\textwidth]{ej41N}
\end{center}


$\mcV$ (kN)
\begin{center}
	\includegraphics[width=.95\textwidth]{ej41V}
\end{center}

$\mcM$ (kN.m)
\begin{center}
	\includegraphics[width=.95\textwidth]{ej41M}
\end{center}


Deformada
\begin{center}
	\includegraphics[width=.95\textwidth]{ej41def}
\end{center}



\item[4.2] 

$\mcN$ (kN)
\begin{center}
	\includegraphics[width=.45\textwidth]{ej42N}
\end{center}


$\mcV$ (kN)
\begin{center}
	\includegraphics[width=.45\textwidth]{ej42V}
\end{center}

$\mcM$ (kN.m)
\begin{center}
	\includegraphics[width=.45\textwidth]{ej42M}
\end{center}

Deformada
\begin{center}
	\includegraphics[width=.45\textwidth]{ej42def}
\end{center}




\item[4.3] 

$\mcN$ (kN)
\begin{center}
	\includegraphics[width=.45\textwidth]{ej43N}
\end{center}


$\mcV$ (kN)
\begin{center}
	\includegraphics[width=.45\textwidth]{ej43V}
\end{center}

$\mcM$ (kN.m)
\begin{center}
	\includegraphics[width=.45\textwidth]{ej43M}
\end{center}

Deformada
\begin{center}
	\includegraphics[width=.45\textwidth]{ej43def}
\end{center}





%
%Ejercicio 4.3
%N (kN)
%
%M (kN∙m)
%
%V (kN)
%
%Deformada
%
%
%Giro de los puntos C, D, E y F, θC = θD = θE = θF = -2,98x10-4.
%Ejercicio 4.4
%a P = 17,5 kN
%b Giro en C: θC = 5,6x10-4.
%Giro en D: θD = -5,6x10-4.
%Desplazamiento de M: ΔM = 2,99x10-3 m (hacia abajo).
%Ejercicio 4.5
%a α1 = 0,457 , α2 = 0,163 , α3 = -0,120.
%
%N (kN)
%
%V (kN)
%
%
%M (kN∙m)
%
%Ejercicio 4.6
%a t = 2,5 cm
%b k = 2000 kN/m
%N (kN)
%
%
%
%V (kN)
%
%M (kN∙m)
%
%Deformada
%
%Ejercicio 4.7
%a k = 1200 kN/m
%b a = 2,82 m
%c σmáx = 6,85 MPa
%
%N (kN)
%
%V (kN)
%
%
%M (kN∙m)
%
%Ejercicio 4.8
%a Giro en C: θC = 3,16x10-3.
%Desplazamiento de C: ΔC = 2,53x10-2 m (hacia la izquierda).
%
%N (kN)
%
%
%M (kN∙m)
%
%V (kN)
%
%
%Deformada
%
%
%Ejercicio 4.9
%N (kN)
%
%
%M (kN∙m)
%
%V (kN)
%
%
%Deformada
%
%
%


\end{description}
